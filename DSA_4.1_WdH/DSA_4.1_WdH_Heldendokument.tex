%%%%%%%%%%%%%%%%%%%%%%%%%%%%%%%%%%%%%%%%%%%%%%%%%%%%%%%%%%%%%%%%%%%%%%%%%%%%%%%%%
% DSA 4.1 Heldendokument by g0hl1n
%   Source:  https://github.com/g0hl1n/DSA_LaTeX_Heldendokument
%   Author:  Richard 'g0hl1n' Leitner <me@g0hl1n.net>
%   Version: 1.0
%   Updated: Thu 03 Apr 2014
%%%%%%%%%%%%%%%%%%%%%%%%%%%%%%%%%%%%%%%%%%%%%%%%%%%%%%%%%%%%%%%%%%%%%%%%%%%%%%%%%
\documentclass[a4paper,10pt]{article}
\usepackage[utf8]{inputenc}
\usepackage[ngerman]{babel}

% Version des Dokuments:
\def \version{1.0}

%----------------------------------------------------------------------------------------
%	PACKAGES
%----------------------------------------------------------------------------------------
\usepackage{geometry} % for page dimensions
\usepackage{fancyhdr} % for custom (fancy) headers and footers
\usepackage{lastpage} % for pagecount
\usepackage{graphicx} % for embedding graphics
\usepackage{hyperref} % for urls and their appearence
\usepackage{color}    % for colored text
\usepackage{graphicx} % Required for including pictures
\usepackage{float}    % Allows putting an [H] in \begin{figure} to specify the exact location of the figure
\usepackage{wrapfig}  % Allows in-line images such as the example fish picture
\usepackage[table,usenames,dvipsnames]{xcolor} % for own color definitions and tables
\usepackage{colortbl}
\usepackage{multirow} % for multirowed columns in tables
\usepackage{booktabs} % table rule magic
\usepackage{array}
\usepackage{amssymb}
\usepackage{xstring} % for some fancy string functions
\usepackage{lscape} % for single landscape pages
\usepackage{forloop}

%----------------------------------------------------------------------------------------
%	KONFIGURATION DES DOKUMENTS (Layout, Farben & Metadaten)
%----------------------------------------------------------------------------------------
\geometry{top=2mm,left=10mm,right=10mm,bottom=2mm,headsep=0mm,footskip=0mm} % Seitenränder

\linespread{1} % Zeilenabstand=1
\graphicspath{{./img/}} % setzt den Pfad zu Grafiken

% Hyperlink Metadata & Link Konfiguration
\hypersetup{
	pdftitle={DSA 4.1 Heldendokument},
	pdfauthor={Richard 'g0hl1n' Leitner},
	pdfsubject={Das Schwarze Auge},
	pdfkeywords={DSA} {Heldendokument} {LaTeX},
	colorlinks,
	citecolor=black,
	filecolor=black,
	linkcolor=black,
	urlcolor=black
}

%-----------------------------------
% Seitenlayout:
\pagestyle{empty} % "Leeres" Seitenlayout (keine Kopf- & Fußzeilen)
\def \footline{DSA-4.1 WdH \LaTeX\ Heldendokument v\version\ by g0hl1n - \url{https://github.com/g0hl1n/DSA_LaTeX_Heldendokument} - \today}

%-----------------------------------
% Farben:
\definecolor{titlepagelinecolor}{HTML}{707070} % Grau

%-----------------------------------
% (globale) Tabellenkonfiguration:
\renewcommand{\arraystretch}{1.4} % Abstand zwischen Zellen

%----------------------------------------------------------------------------------------
%	DEFINITION EIGENER KOMMANDOS
%----------------------------------------------------------------------------------------
% VRule
\newcommand\VRule[1][\arrayrulewidth]{\vrule width #1}

% Cline ... eine cline mit konfigurierbarer Dicke
\newlength{\Oldarrayrulewidth}
\newcommand{\Cline}[2]{%
  \noalign{\global\setlength{\Oldarrayrulewidth}{\arrayrulewidth}}%
  \noalign{\global\setlength{\arrayrulewidth}{#1}}\cline{#2}%
  \noalign{\global\setlength{\arrayrulewidth}{\Oldarrayrulewidth}}%
}

\makeatletter

%----------------------------
%	Zauberzeug
\newcounter{ZauberAnzahl}
\def\ZauberTabelleInhaltA{}
\def\ZauberTabelleInhaltB{}

% neuen Zauber hinzufügen
% Argumente sind:
%	 1: Zaubername
%	 2: Probe Eig. 1
%	 3: Probe Eig. 2
%	 4: Probe Eig. 3
%	 5: ZfW
%	 6: ZD
%	 7: Kosten
%	 8: Reichweite
%	 9: WD
%	10: Anmerkungen
%	11: Komplexität
%	12: Repräsentation
%	13: Hauszauber?
%	14: Lern
\newcommand{\neuerZauber}[9]{%
\addtocounter{ZauberAnzahl}{1}
\ifnum\value{ZauberAnzahl}>32\g@addto@macro\ZauberTabelleInhaltB{ & #1 & #2 $\bullet$ #3 $\bullet$ #4 & #5 & & #6 & #7 & #8 & #9 &}
\else\g@addto@macro\ZauberTabelleInhaltA{ & #1 & #2 $\bullet$ #3 $\bullet$ #4 & #5 & & #6 & #7 & #8 & #9 &}
\fi
\neuerZauberCont
}
\newcommand{\neuerZauberCont}[5]{%
\ifnum\value{ZauberAnzahl}>32\g@addto@macro\ZauberTabelleInhaltB{ #1 & #2 & #3 & #4 & #5\\\hline}
\else\g@addto@macro\ZauberTabelleInhaltA{ #1 & #2 & #3 & #4 & #5\\\hline}
\fi
}

\makeatother
 % Commandset laden
%----------------------------------------------------------------------------------------
%	BEGINN DES DOKUMENTS
%----------------------------------------------------------------------------------------
%----------------------------
%	Informationen
\def \CharakterName{}
\def \CharakterRasse{}
\def \CharakterKultur{}
\def \CharakterProfession{}
\def \CharakterPortraitBild{Wappen_Leer.png}

\def \CharakterGPBasis{110}
\def \CharakterRasseModifikationen{}
\def \CharakterKulturModifikationen{}
\def \CharakterProfessionModifikationen{}

\def \CharakterGeschlecht{}
\def \CharakterAlter{}
\def \CharakterGroesse{ Schritt}
\def \CharakterGewicht{ Stein}
\def \CharakterHaarfarbe{}
\def \CharakterAugenfarbe{}

\def \CharakterAussehenA{}
\def \CharakterAussehenB{}
\def \CharakterAussehenC{}
\def \CharakterAussehenD{}

\def \CharakterStand{}
\def \CharakterTitel{}
\def \CharakterSO{}

\def \CharakterHintergrundA{}
\def \CharakterHintergrundB{}
\def \CharakterHintergrundC{}
\def \CharakterHintergrundD{}
\def \CharakterHintergrundE{}
\def \CharakterHintergrundF{}

% Charakternotizen
\def \CharakterNotizGenerationsdatum{01.01.1970}
\def \CharakterNotizLetzteAenderung{01.01.1970}
\def \CharakterNotizAnzahlAbenteuer{0}
\def \CharakterNotizKompilierungsDatum{\today}

%----------------------------
%	Vor- und Nachteile
\def \CharakterVorNachteileA{} % 1. Zeile
\def \CharakterVorNachteileB{} % 2. Zeile
\def \CharakterVorNachteileC{} % 3. Zeile
\def \CharakterVorNachteileD{} % 4. Zeile
\def \CharakterVorNachteileE{} % 5. Zeile

%----------------------------
%	Eigenschaften
% Mut
\def \EigMUmod{0}
\def \EigMUstart{0}
\def \EigMUaktuell{0}
% Klugheit
\def \EigKLmod{0}
\def \EigKLstart{0}
\def \EigKLaktuell{0}
% Intuition
\def \EigINmod{0}
\def \EigINstart{0}
\def \EigINaktuell{0}
% Charisma
\def \EigCHmod{0}
\def \EigCHstart{0}
\def \EigCHaktuell{0}
% Fingerfertigkeit
\def \EigFFmod{0}
\def \EigFFstart{0}
\def \EigFFaktuell{0}
% Gewandtheit
\def \EigGEmod{0}
\def \EigGEstart{0}
\def \EigGEaktuell{0}
% Konstitution
\def \EigKOmod{0}
\def \EigKOstart{0}
\def \EigKOaktuell{0}
% Körperkraft
\def \EigKKmod{0}
\def \EigKKstart{0}
\def \EigKKaktuell{0}
% Geschwindigkeit
\def \EigGSmod{0}
\def \EigGSstart{0}
\def \EigGSaktuell{0}

%----------------------------
%	Basiswerte
% Lebenspunkte
\def \BasisLEmod{0}
\def \BasisLEstart{0}
\def \BasisLEaktuell{0}
\def \BasisLEaktuellHaelfte{0}
\def \BasisLEaktuellDrittel{0}
\def \BasisLEaktuellViertel{0}
\def \BasisLEzugekauft{0}
\def \BasisLEmaxZugekauft{0}
% Ausdauer
\def \BasisAUmod{0}
\def \BasisAUstart{0}
\def \BasisAUaktuell{0}
\def \BasisAUaktuellHaelfte{0}
\def \BasisAUaktuellDrittel{0}
\def \BasisAUaktuellViertel{0}
\def \BasisAUzugekauft{0}
\def \BasisAUmaxZugekauft{0}
% Astralenergie
\def \BasisAEmod{0}
\def \BasisAEstart{0}
\def \BasisAEaktuell{0}
\def \BasisAEzugekauft{0}
\def \BasisAEmaxZugekauft{0}
% Karmaenergie
\def \BasisKEmod{0}
\def \BasisKEstart{0}
\def \BasisKEaktuell{0}
\def \BasisKEzugekauft{0}
\def \BasisKEmaxZugekauft{0}
% Magieresistenz
\def \BasisMRmod{0}
\def \BasisMRstart{0}
\def \BasisMRaktuell{0}
\def \BasisMRzugekauft{0}
\def \BasisMRmaxZugekauft{0}
% INI-Basiswert
\def \BasisINImod{0}
\def \BasisINIstart{0}
\def \BasisINIaktuell{0}
% AT-Basiswert
\def \BasisATmod{0}
\def \BasisATstart{0}
\def \BasisATaktuell{0}
% PA-Basiswert
\def \BasisPAmod{0}
\def \BasisPAstart{0}
\def \BasisPAaktuell{0}
% FK-Basiswert
\def \BasisFKmod{0}
\def \BasisFKstart{0}
\def \BasisFKaktuell{0}
% Wundschwelle
\def \BasisWundschwelle{0}

%----------------------------
%	Abenteuerpunkte
\def \APgesamt{0}
\def \APeingesetzt{0}
\def \APguthaben{0}

%----------------------------
%	Sonderfertigkeiten
% Ausser Kampf
\def \SFausserKampfA{}
\def \SFausserKampfB{}
\def \SFausserKampfC{}
\def \SFausserKampfD{}
\def \SFausserKampfE{}
\def \SFausserKampfF{}

%----------------------------
%	Gaben
\def \GabeA{} % 1. Gabe:
\def \GabeATaWL{}
\def \GabeATaWR{}
\def \GabeB{} % 2. Gabe:
\def \GabeBTaWL{}
\def \GabeBTaWR{}
\def \GabeC{} % 3. Gabe:
\def \GabeCTaWL{}
\def \GabeCTaWR{}

%----------------------------
%	Talente
% Basis Kampftechniken
\def \TalentDolcheAT{0} % Dolche
\def \TalentDolchePA{0}
\def \TalentDolcheTaWL{0}
\def \TalentDolcheTaWR{}
\def \TalentHiebwaffenAT{0} % Hiebwaffen
\def \TalentHiebwaffenPA{0}
\def \TalentHiebwaffenTaWL{0}
\def \TalentHiebwaffenTaWR{}
\def \TalentRaufenAT{0} % Raufen
\def \TalentRaufenPA{0}
\def \TalentRaufenTaWL{0}
\def \TalentRaufenTaWR{}
\def \TalentRingenAT{0} % Ringen
\def \TalentRingenPA{0}
\def \TalentRingenTaWL{0}
\def \TalentRingenTaWR{}
\def \TalentSaebelAT{0} % Säbel
\def \TalentSaebelPA{0}
\def \TalentSaebelTaWL{0}
\def \TalentSaebelTaWR{}
\def \TalentWurfmesserAT{0} % Wurfmesser
\def \TalentWurfmesserPA{0}
\def \TalentWurfmesserTaWL{0}
\def \TalentWurfmesserTaWR{}

% Extra Kampftechniken
\def \TalentKampfExtraA{} % 1. extra Kampftechnik:
\def \TalentKampfExtraASKat{}
\def \TalentKampfExtraABE{}
\def \TalentKampfExtraAAT{}
\def \TalentKampfExtraAPA{}
\def \TalentKampfExtraATaWL{}
\def \TalentKampfExtraATaWR{}
\def \TalentKampfExtraB{} % 2. extra Kampftechnik:
\def \TalentKampfExtraBSKat{}
\def \TalentKampfExtraBBE{}
\def \TalentKampfExtraBAT{}
\def \TalentKampfExtraBPA{}
\def \TalentKampfExtraBTaWL{}
\def \TalentKampfExtraBTaWR{}
\def \TalentKampfExtraC{} % 3. extra Kampftechnik:
\def \TalentKampfExtraCSKat{}
\def \TalentKampfExtraCBE{}
\def \TalentKampfExtraCAT{}
\def \TalentKampfExtraCPA{}
\def \TalentKampfExtraCTaWL{}
\def \TalentKampfExtraCTaWR{}
\def \TalentKampfExtraD{} % 4. extra Kampftechnik:
\def \TalentKampfExtraDSKat{}
\def \TalentKampfExtraDBE{}
\def \TalentKampfExtraDAT{}
\def \TalentKampfExtraDPA{}
\def \TalentKampfExtraDTaWL{}
\def \TalentKampfExtraDTaWR{}
\def \TalentKampfExtraE{} % 5. extra Kampftechnik:
\def \TalentKampfExtraESKat{}
\def \TalentKampfExtraEBE{}
\def \TalentKampfExtraEAT{}
\def \TalentKampfExtraEPA{}
\def \TalentKampfExtraETaWL{}
\def \TalentKampfExtraETaWR{}
\def \TalentKampfExtraF{} % 6. extra Kampftechnik:
\def \TalentKampfExtraFSKat{}
\def \TalentKampfExtraFBE{}
\def \TalentKampfExtraFAT{}
\def \TalentKampfExtraFPA{}
\def \TalentKampfExtraFTaWL{}
\def \TalentKampfExtraFTaWR{}
\def \TalentKampfExtraG{} % 7. extra Kampftechnik:
\def \TalentKampfExtraGSKat{}
\def \TalentKampfExtraGBE{}
\def \TalentKampfExtraGAT{}
\def \TalentKampfExtraGPA{}
\def \TalentKampfExtraGTaWL{}
\def \TalentKampfExtraGTaWR{}
\def \TalentKampfExtraH{} % 8. extra Kampftechnik:
\def \TalentKampfExtraHSKat{}
\def \TalentKampfExtraHBE{}
\def \TalentKampfExtraHAT{}
\def \TalentKampfExtraHPA{}
\def \TalentKampfExtraHTaWL{}
\def \TalentKampfExtraHTaWR{}
\def \TalentKampfExtraI{} % 9. extra Kampftechnik:
\def \TalentKampfExtraISKat{}
\def \TalentKampfExtraIBE{}
\def \TalentKampfExtraIAT{}
\def \TalentKampfExtraIPA{}
\def \TalentKampfExtraITaWL{}
\def \TalentKampfExtraITaWR{}
\def \TalentKampfExtraJ{} % 10. extra Kampftechnik
\def \TalentKampfExtraJSKat{}
\def \TalentKampfExtraJBE{}
\def \TalentKampfExtraJAT{}
\def \TalentKampfExtraJPA{}
\def \TalentKampfExtraJTaWL{}
\def \TalentKampfExtraJTaWR{}

% Basis Körperliche Talente
\def \TalentAthletikTaWL{0} % Athletik
\def \TalentAthletikTaWR{}
\def \TalentKletternTaWL{0} % Klettern
\def \TalentKletternTaWR{}
\def \TalentKoerperbeherrschungTaWL{0} % Körperbeherrschung
\def \TalentKoerperbeherrschungTaWR{}
\def \TalentSchleichenTaWL{0} % Schleichen
\def \TalentSchleichenTaWR{}
\def \TalentSchwimmenTaWL{0} % Schwimmen
\def \TalentSchwimmenTaWR{}
\def \TalentSelbstbeherrschungTaWL{0} % Selbstbeherrschung
\def \TalentSelbstbeherrschungTaWR{}
\def \TalentSichVersteckenTaWL{0} % Sich Verstecken
\def \TalentSichVersteckenTaWR{}
\def \TalentSingenTaWL{0} % Singen
\def \TalentSingenTaWR{}
\def \TalentSinnesschaerfeTaWL{0} % Sinnesschärfe
\def \TalentSinnesschaerfeTaWR{}
\def \TalentTanzenTaWL{0} % Tanzen
\def \TalentTanzenTaWR{}
\def \TalentZechenTaWL{0} % Zechen
\def \TalentZechenTaWR{}

% Extra Körperliche Talente
\def \TalentKoerperExtraA{} % 1. extra körperliches talent:
\def \TalentKoerperExtraAProbe{}
\def \TalentKoerperExtraABE{}
\def \TalentKoerperExtraATaWL{}
\def \TalentKoerperExtraATaWR{}
\def \TalentKoerperExtraB{} % 2. extra körperliches talent:
\def \TalentKoerperExtraBProbe{}
\def \TalentKoerperExtraBBE{}
\def \TalentKoerperExtraBTaWL{}
\def \TalentKoerperExtraBTaWR{}
\def \TalentKoerperExtraC{} % 3. extra körperliches talent:
\def \TalentKoerperExtraCProbe{}
\def \TalentKoerperExtraCBE{}
\def \TalentKoerperExtraCTaWL{}
\def \TalentKoerperExtraCTaWR{}
\def \TalentKoerperExtraD{} % 4. extra körperliches talent:
\def \TalentKoerperExtraDProbe{}
\def \TalentKoerperExtraDBE{}
\def \TalentKoerperExtraDTaWL{}
\def \TalentKoerperExtraDTaWR{}
\def \TalentKoerperExtraE{} % 5. extra körperliches talent:
\def \TalentKoerperExtraEProbe{}
\def \TalentKoerperExtraEBE{}
\def \TalentKoerperExtraETaWL{}
\def \TalentKoerperExtraETaWR{}

% Basis Gesellschaftliche Talente
\def \TalentMenschenkenntnisTaWL{0} % Menschenkenntnis
\def \TalentMenschenkenntnisTaWR{}
\def \TalentUeberredenTaWL{0} % Überreden
\def \TalentUeberredenTaWR{}

% Extra Gesellschaftliche Talente
\def \TalentGesellschaftExtraA{} % 1. extra gesellschaftliches Talent:
\def \TalentGesellschaftExtraAProbe{}
\def \TalentGesellschaftExtraATaWL{}
\def \TalentGesellschaftExtraATaWR{}
\def \TalentGesellschaftExtraB{} % 2. extra gesellschaftliches Talent:
\def \TalentGesellschaftExtraBProbe{}
\def \TalentGesellschaftExtraBTaWL{}
\def \TalentGesellschaftExtraBTaWR{}
\def \TalentGesellschaftExtraC{} % 3. extra gesellschaftliches Talent:
\def \TalentGesellschaftExtraCProbe{}
\def \TalentGesellschaftExtraCTaWL{}
\def \TalentGesellschaftExtraCTaWR{}
\def \TalentGesellschaftExtraD{} % 4. extra gesellschaftliches Talent:
\def \TalentGesellschaftExtraDProbe{}
\def \TalentGesellschaftExtraDTaWL{}
\def \TalentGesellschaftExtraDTaWR{}
\def \TalentGesellschaftExtraE{} % 5. extra gesellschaftliches Talent:
\def \TalentGesellschaftExtraEProbe{}
\def \TalentGesellschaftExtraETaWL{}
\def \TalentGesellschaftExtraETaWR{}

% Basis Natur-Talente
\def \TalentFaehrtensuchenTaWL{0} % Fährtensuchen
\def \TalentFaehrtensuchenTaWR{}
\def \TalentOrientierungTaWL{0} % Orientierung
\def \TalentOrientierungTaWR{}
\def \TalentWildnislebenTaWL{0} % Wildnisleben
\def \TalentWildnislebenTaWR{}

% Extra Natur-Talente
\def \TalentNaturExtraA{} % 1. extra Natur-Talent:
\def \TalentNaturExtraAProbe{}
\def \TalentNaturExtraATaWL{}
\def \TalentNaturExtraATaWR{}
\def \TalentNaturExtraB{} % 2. extra Natur-Talent:
\def \TalentNaturExtraBProbe{}
\def \TalentNaturExtraBTaWL{}
\def \TalentNaturExtraBTaWR{}
\def \TalentNaturExtraC{} % 3. extra Natur-Talent:
\def \TalentNaturExtraCProbe{}
\def \TalentNaturExtraCTaWL{}
\def \TalentNaturExtraCTaWR{}
\def \TalentNaturExtraD{} % 4. extra Natur-Talent:
\def \TalentNaturExtraDProbe{}
\def \TalentNaturExtraDTaWL{}
\def \TalentNaturExtraDTaWR{}

% Basis Wissenstalente
\def \TalentGoetterKulteTaWL{0} % Götter/Kulte
\def \TalentGoetterKulteTaWR{}
\def \TalentRechnenTaWL{0} % Rechnen
\def \TalentRechnenTaWR{}
\def \TalentSagenLegendenTaWL{0} % Sagen/Legenden
\def \TalentSagenLegendenTaWR{}

% Extra Wissenstalente
\def \TalentWissenExtraA{} % 1. extra Wissenstalent:
\def \TalentWissenExtraAProbe{}
\def \TalentWissenExtraATaWL{}
\def \TalentWissenExtraATaWR{}
\def \TalentWissenExtraB{} % 2. extra Wissenstalent:
\def \TalentWissenExtraBProbe{}
\def \TalentWissenExtraBTaWL{}
\def \TalentWissenExtraBTaWR{}
\def \TalentWissenExtraC{} % 3. extra Wissenstalent:
\def \TalentWissenExtraCProbe{}
\def \TalentWissenExtraCTaWL{}
\def \TalentWissenExtraCTaWR{}
\def \TalentWissenExtraD{} % 4. extra Wissenstalent:
\def \TalentWissenExtraDProbe{}
\def \TalentWissenExtraDTaWL{}
\def \TalentWissenExtraDTaWR{}
\def \TalentWissenExtraE{} % 5. extra Wissenstalent:
\def \TalentWissenExtraEProbe{}
\def \TalentWissenExtraETaWL{}
\def \TalentWissenExtraETaWR{}
\def \TalentWissenExtraF{} % 6. extra Wissenstalent:
\def \TalentWissenExtraFProbe{}
\def \TalentWissenExtraFTaWL{}
\def \TalentWissenExtraFTaWR{}
\def \TalentWissenExtraG{} % 7. extra Wissenstalent:
\def \TalentWissenExtraGProbe{}
\def \TalentWissenExtraGTaWL{}
\def \TalentWissenExtraGTaWR{}
\def \TalentWissenExtraH{} % 8. extra Wissenstalent:
\def \TalentWissenExtraHProbe{}
\def \TalentWissenExtraHTaWL{}
\def \TalentWissenExtraHTaWR{}
\def \TalentWissenExtraI{} % 9. extra Wissenstalent:
\def \TalentWissenExtraIProbe{}
\def \TalentWissenExtraITaWL{}
\def \TalentWissenExtraITaWR{}
\def \TalentWissenExtraJ{} % 10. extra Wissenstalent:
\def \TalentWissenExtraJProbe{}
\def \TalentWissenExtraJTaWL{}
\def \TalentWissenExtraJTaWR{}
\def \TalentWissenExtraK{} % 11. extra Wissenstalent:
\def \TalentWissenExtraKProbe{}
\def \TalentWissenExtraKTaWL{}
\def \TalentWissenExtraKTaWR{}
\def \TalentWissenExtraL{} % 12. extra Wissenstalent:
\def \TalentWissenExtraLProbe{}
\def \TalentWissenExtraLTaWL{}
\def \TalentWissenExtraLTaWR{}
\def \TalentWissenExtraM{} % 13. extra Wissenstalent:
\def \TalentWissenExtraMProbe{}
\def \TalentWissenExtraMTaWL{}
\def \TalentWissenExtraMTaWR{}
\def \TalentWissenExtraN{} % 14. extra Wissenstalent:
\def \TalentWissenExtraNProbe{}
\def \TalentWissenExtraNTaWL{}
\def \TalentWissenExtraNTaWR{}

% Sprachen & Schriften
\def \TalentMuttersprache{} % Muttersprache:
\def \TalentMutterspracheKomp{} % Komplexität
\def \TalentMutterspracheTaWL{}
\def \TalentMutterspracheTaWR{}
\def \TalentSpracheSchriftExtraA{} % 1. extra Sprache/Schrift:
\def \TalentSpracheSchriftExtraAKomp{}
\def \TalentSpracheSchriftExtraATaWL{}
\def \TalentSpracheSchriftExtraATaWR{}
\def \TalentSpracheSchriftExtraB{} % 2. extra Sprache/Schrift:
\def \TalentSpracheSchriftExtraBKomp{}
\def \TalentSpracheSchriftExtraBTaWL{}
\def \TalentSpracheSchriftExtraBTaWR{}
\def \TalentSpracheSchriftExtraC{} % 3. extra Sprache/Schrift:
\def \TalentSpracheSchriftExtraCKomp{}
\def \TalentSpracheSchriftExtraCTaWL{}
\def \TalentSpracheSchriftExtraCTaWR{}
\def \TalentSpracheSchriftExtraD{} % 4. extra Sprache/Schrift:
\def \TalentSpracheSchriftExtraDKomp{}
\def \TalentSpracheSchriftExtraDTaWL{}
\def \TalentSpracheSchriftExtraDTaWR{}
\def \TalentSpracheSchriftExtraE{} % 5. extra Sprache/Schrift:
\def \TalentSpracheSchriftExtraEKomp{}
\def \TalentSpracheSchriftExtraETaWL{}
\def \TalentSpracheSchriftExtraETaWR{}
\def \TalentSpracheSchriftExtraF{} % 6. extra Sprache/Schrift:
\def \TalentSpracheSchriftExtraFKomp{}
\def \TalentSpracheSchriftExtraFTaWL{}
\def \TalentSpracheSchriftExtraFTaWR{}
\def \TalentSpracheSchriftExtraG{} % 7. extra Sprache/Schrift:
\def \TalentSpracheSchriftExtraGKomp{}
\def \TalentSpracheSchriftExtraGTaWL{}
\def \TalentSpracheSchriftExtraGTaWR{}
\def \TalentSpracheSchriftExtraH{} % 8. extra Sprache/Schrift:
\def \TalentSpracheSchriftExtraHKomp{}
\def \TalentSpracheSchriftExtraHTaWL{}
\def \TalentSpracheSchriftExtraHTaWR{}
\def \TalentSpracheSchriftExtraI{} % 9. extra Sprache/Schrift:
\def \TalentSpracheSchriftExtraIKomp{}
\def \TalentSpracheSchriftExtraITaWL{}
\def \TalentSpracheSchriftExtraITaWR{}
\def \TalentSpracheSchriftExtraJ{} % 10. extra Sprache/Schrift:
\def \TalentSpracheSchriftExtraJKomp{}
\def \TalentSpracheSchriftExtraJTaWL{}
\def \TalentSpracheSchriftExtraJTaWR{}

% Basis  Handwerkliche Talente
\def \TalentHeilkundeWundenTaWL{0} % Heilkunde Wunden
\def \TalentHeilkundeWundenTaWR{}
\def \TalentHolzbearbeitungTaWL{0} % Holzbearbeitung
\def \TalentHolzbearbeitungTaWR{}
\def \TalentKochenTaWL{0} % Kochen
\def \TalentKochenTaWR{}
\def \TalentLederarbeitenTaWL{0} % Lederarbeiten
\def \TalentLederarbeitenTaWR{}
\def \TalentMalenZeichnenTaWL{0} % Malen/Zeichnen
\def \TalentMalenZeichnenTaWR{}
\def \TalentSchneidernTaWL{0} % Schneidern
\def \TalentSchneidernTaWR{}

% Extra Handwerkliche Talente
\def \TalentHandwerkExtraA{} % 1. extra handwerkliches Talent:
\def \TalentHandwerkExtraAProbe{}
\def \TalentHandwerkExtraATaWL{}
\def \TalentHandwerkExtraATaWR{}
\def \TalentHandwerkExtraB{} % 2. extra handwerkliches Talent:
\def \TalentHandwerkExtraBProbe{}
\def \TalentHandwerkExtraBTaWL{}
\def \TalentHandwerkExtraBTaWR{}
\def \TalentHandwerkExtraC{} % 3. extra handwerkliches Talent:
\def \TalentHandwerkExtraCProbe{}
\def \TalentHandwerkExtraCTaWL{}
\def \TalentHandwerkExtraCTaWR{}
\def \TalentHandwerkExtraD{} % 4. extra handwerkliches Talent:
\def \TalentHandwerkExtraDProbe{}
\def \TalentHandwerkExtraDTaWL{}
\def \TalentHandwerkExtraDTaWR{}
\def \TalentHandwerkExtraE{} % 5. extra handwerkliches Talent:
\def \TalentHandwerkExtraEProbe{}
\def \TalentHandwerkExtraETaWL{}
\def \TalentHandwerkExtraETaWR{}
\def \TalentHandwerkExtraF{} % 6. extra handwerkliches Talent:
\def \TalentHandwerkExtraFProbe{}
\def \TalentHandwerkExtraFTaWL{}
\def \TalentHandwerkExtraFTaWR{}
\def \TalentHandwerkExtraG{} % 7. extra handwerkliches Talent:
\def \TalentHandwerkExtraGProbe{}
\def \TalentHandwerkExtraGTaWL{}
\def \TalentHandwerkExtraGTaWR{}
\def \TalentHandwerkExtraH{} % 8. extra handwerkliches Talent:
\def \TalentHandwerkExtraHProbe{}
\def \TalentHandwerkExtraHTaWL{}
\def \TalentHandwerkExtraHTaWR{}
\def \TalentHandwerkExtraI{} % 9. extra handwerkliches Talent:
\def \TalentHandwerkExtraIProbe{}
\def \TalentHandwerkExtraITaWL{}
\def \TalentHandwerkExtraITaWR{}
\def \TalentHandwerkExtraJ{} % 10. extra handwerkliches Talent:
\def \TalentHandwerkExtraJProbe{}
\def \TalentHandwerkExtraJTaWL{}
\def \TalentHandwerkExtraJTaWR{}

%----------------------------
%	Waffen & Kampfwerte
% Nahkampf
\def \WaffeNahkampfA{} % 1. Nahkampfwaffe
\def \WaffeNahkampfATypeBE{} % Typ/eBE
\def \WaffeNahkampfADK{} % DK
\def \WaffeNahkampfATP{} % TP
\def \WaffeNahkampfATPKK{} % TP/KK
\def \WaffeNahkampfAINI{} % INI
\def \WaffeNahkampfAWM{} % WM
\def \WaffeNahkampfAAT{} % AT
\def \WaffeNahkampfAPA{} % PA
\def \WaffeNahkampfABFa{} % Bruchfaktor: 1. Spalte
\def \WaffeNahkampfABFb{} % Bruchfaktor: 2. Spalte
\def \WaffeNahkampfABFc{} % Bruchfaktor: 3. Spalte
\def \WaffeNahkampfABFd{} % Bruchfaktor: 4. Spalte
\def \WaffeNahkampfABFe{} % Bruchfaktor: 5. Spalte

\def \WaffeNahkampfB{} % 2. Nahkampfwaffe
\def \WaffeNahkampfBTypeBE{} % Typ/eBE
\def \WaffeNahkampfBDK{} % DK
\def \WaffeNahkampfBTP{} % TP
\def \WaffeNahkampfBTPKK{} % TP/KK
\def \WaffeNahkampfBINI{} % INI
\def \WaffeNahkampfBWM{} % WM
\def \WaffeNahkampfBAT{} % AT
\def \WaffeNahkampfBPA{} % PA
\def \WaffeNahkampfBBFa{} % Bruchfaktor: 1. Spalte
\def \WaffeNahkampfBBFb{} % Bruchfaktor: 2. Spalte
\def \WaffeNahkampfBBFc{} % Bruchfaktor: 3. Spalte
\def \WaffeNahkampfBBFd{} % Bruchfaktor: 4. Spalte
\def \WaffeNahkampfBBFe{} % Bruchfaktor: 5. Spalte

\def \WaffeNahkampfC{} % 3. Nahkampfwaffe
\def \WaffeNahkampfCTypeBE{} % Typ/eBE
\def \WaffeNahkampfCDK{} % DK
\def \WaffeNahkampfCTP{} % TP
\def \WaffeNahkampfCTPKK{} % TP/KK
\def \WaffeNahkampfCINI{} % INI
\def \WaffeNahkampfCWM{} % WM
\def \WaffeNahkampfCAT{} % AT
\def \WaffeNahkampfCPA{} % PA
\def \WaffeNahkampfCBFa{} % Bruchfaktor: 1. Spalte
\def \WaffeNahkampfCBFb{} % Bruchfaktor: 2. Spalte
\def \WaffeNahkampfCBFc{} % Bruchfaktor: 3. Spalte
\def \WaffeNahkampfCBFd{} % Bruchfaktor: 4. Spalte
\def \WaffeNahkampfCBFe{} % Bruchfaktor: 5. Spalte

\def \WaffeNahkampfD{} % 4. Nahkampfwaffe
\def \WaffeNahkampfDTypeBE{} % Typ/eBE
\def \WaffeNahkampfDDK{} % DK
\def \WaffeNahkampfDTP{} % TP
\def \WaffeNahkampfDTPKK{} % TP/KK
\def \WaffeNahkampfDINI{} % INI
\def \WaffeNahkampfDWM{} % WM
\def \WaffeNahkampfDAT{} % AT
\def \WaffeNahkampfDPA{} % PA
\def \WaffeNahkampfDBFa{} % Bruchfaktor: 1. Spalte
\def \WaffeNahkampfDBFb{} % Bruchfaktor: 2. Spalte
\def \WaffeNahkampfDBFc{} % Bruchfaktor: 3. Spalte
\def \WaffeNahkampfDBFd{} % Bruchfaktor: 4. Spalte
\def \WaffeNahkampfDBFe{} % Bruchfaktor: 5. Spalte

\def \WaffeNahkampfE{} % 5. Nahkampfwaffe
\def \WaffeNahkampfETypeBE{} % Typ/eBE
\def \WaffeNahkampfEDK{} % DK
\def \WaffeNahkampfETP{} % TP
\def \WaffeNahkampfETPKK{} % TP/KK
\def \WaffeNahkampfEINI{} % INI
\def \WaffeNahkampfEWM{} % WM
\def \WaffeNahkampfEAT{} % AT
\def \WaffeNahkampfEPA{} % PA
\def \WaffeNahkampfEBFa{} % Bruchfaktor: 1. Spalte
\def \WaffeNahkampfEBFb{} % Bruchfaktor: 2. Spalte
\def \WaffeNahkampfEBFc{} % Bruchfaktor: 3. Spalte
\def \WaffeNahkampfEBFd{} % Bruchfaktor: 4. Spalte
\def \WaffeNahkampfEBFe{} % Bruchfaktor: 5. Spalte

\def \SonderfertigkeitenNahkampfA{} % Nahkampf Sonderfertigkeiten (1. Zeile)
\def \SonderfertigkeitenNahkampfB{} % Nahkampf Sonderfertigkeiten (2. Zeile)
\def \SonderfertigkeitenNahkampfC{} % Nahkampf Sonderfertigkeiten (3. Zeile)

% Fernkampf
\def \WaffeFernkampfA{} % 1. Fernkampfwaffe
\def \WaffeFernkampfATypeBE{} % Typ/eBE
\def \WaffeFernkampfATP{} % TP
\def \WaffeFernkampfAEntfernung{} % Entfernungen
\def \WaffeFernkampfATPEntfernung{} % TP/Entfernungen
\def \WaffeFernkampfAFK{} % FK
\def \WaffeFernkampfAGeschosseA{} % Geschosse: 1. Spalte
\def \WaffeFernkampfAGeschosseB{} % Geschosse: 2. Spalte
\def \WaffeFernkampfAGeschosseC{} % Geschosse: 3. Spalte
\def \WaffeFernkampfAGeschosseD{} % Geschosse: 4. Spalte
\def \WaffeFernkampfAGeschosseE{} % Geschosse: 5. Spalte
\def \WaffeFernkampfAGeschosseF{} % Geschosse: 6. Spalte
\def \WaffeFernkampfAGeschosseG{} % Geschosse: 7. Spalte
\def \WaffeFernkampfAGeschosseH{} % Geschosse: 8. Spalte
\def \WaffeFernkampfAGeschosseI{} % Geschosse: 9. Spalte

\def \WaffeFernkampfB{} % 2. Fernkampfwaffe
\def \WaffeFernkampfBTypeBE{}
\def \WaffeFernkampfBTP{}
\def \WaffeFernkampfBEntfernung{}
\def \WaffeFernkampfBTPEntfernung{}
\def \WaffeFernkampfBFK{}
\def \WaffeFernkampfBGeschosseA{}
\def \WaffeFernkampfBGeschosseB{}
\def \WaffeFernkampfBGeschosseC{}
\def \WaffeFernkampfBGeschosseD{}
\def \WaffeFernkampfBGeschosseE{}
\def \WaffeFernkampfBGeschosseF{}
\def \WaffeFernkampfBGeschosseG{}
\def \WaffeFernkampfBGeschosseH{}
\def \WaffeFernkampfBGeschosseI{}

\def \WaffeFernkampfC{} % 3. Fernkampfwaffe
\def \WaffeFernkampfCTypeBE{}
\def \WaffeFernkampfCTP{}
\def \WaffeFernkampfCEntfernung{}
\def \WaffeFernkampfCTPEntfernung{}
\def \WaffeFernkampfCFK{}
\def \WaffeFernkampfCGeschosseA{}
\def \WaffeFernkampfCGeschosseB{}
\def \WaffeFernkampfCGeschosseC{}
\def \WaffeFernkampfCGeschosseD{}
\def \WaffeFernkampfCGeschosseE{}
\def \WaffeFernkampfCGeschosseF{}
\def \WaffeFernkampfCGeschosseG{}
\def \WaffeFernkampfCGeschosseH{}
\def \WaffeFernkampfCGeschosseI{}

\def \SonderfertigkeitenFernkampfA{} % Fernkampf Sonderfertigkeiten (1. Zeile)
\def \SonderfertigkeitenFernkampfB{} % Fernkampf Sonderfertigkeiten (2. Zeile)

% Waffenloser Kampf
\def \WaffenlosRaufenTPKK{} %Raufen
\def \WaffenlosRaufenINI{}
\def \WaffenlosRaufenAT{}
\def \WaffenlosRaufenPA{}
\def \WaffenlosRaufenTPA{}

\def \WaffenlosRingenTPKK{} %Ringen
\def \WaffenlosRingenINI{}
\def \WaffenlosRingenAT{}
\def \WaffenlosRaufenPA{}
\def \WaffenlosRaufenTPA{}

\def \SonderfertigkeitenWaffeloserKampfA{} % Sonderfertigkeiten/Manöver Waffenloser Kampf (1. Zeile)
\def \SonderfertigkeitenWaffeloserKampfB{} % Sonderfertigkeiten/Manöver Waffenloser Kampf (2. Zeile)
\def \SonderfertigkeitenWaffeloserKampfC{} % Sonderfertigkeiten/Manöver Waffenloser Kampf (3. Zeile)

% Schild/Parierwaffen
\def \WaffeSchildA{} % 1. Schild/Parierwaffe
\def \WaffeSchildATyp{} % Typ
\def \WaffeSchildAINI{} % INI
\def \WaffeSchildAWM{}  % WM
\def \WaffeSchildAPA{}  % PA
\def \WaffeSchildABFa{} % Bruchfaktor: 1. Spalte
\def \WaffeSchildABFb{} % Bruchfaktor: 2. Spalte
\def \WaffeSchildABFc{} % Bruchfaktor: 3. Spalte
\def \WaffeSchildABFd{} % Bruchfaktor: 4. Spalte
\def \WaffeSchildABFe{} % Bruchfaktor: 5. Spalte

\def \WaffeSchildB{} % 2. Schild/Parierwaffe
\def \WaffeSchildBTyp{}
\def \WaffeSchildBINI{}
\def \WaffeSchildBWM{}
\def \WaffeSchildBPA{}
\def \WaffeSchildBBFa{}
\def \WaffeSchildBBFb{}
\def \WaffeSchildBBFc{}
\def \WaffeSchildBBFd{}
\def \WaffeSchildBBFe{}

% Rüstung
\def \RuestungA{}
\def \RuestungARS{}
\def \RuestungABE{}
\def \RuestungB{}
\def \RuestungBRS{}
\def \RuestungBBE{}
\def \RuestungC{}
\def \RuestungCRS{}
\def \RuestungCBE{}
\def \RuestungD{}
\def \RuestungDRS{}
\def \RuestungDBE{}
\def \RuestungE{}
\def \RuestungERS{}
\def \RuestungEBE{}

\def \RuestungSummeRS{}
\def \RuestungSummeBE{}
\def \RuestungsgewoehnungWert{}

%----------------------------
%	Ausweichen
\def \AusweichenPA{0}
\def \AusweichenBE{0}
\def \AusweichenSumme{0}

%----------------------------
%	Ausrüstung
% Kleidung
\def \KleidungA{}
\def \KleidungB{}
\def \KleidungC{}
\def \KleidungD{}
\def \KleidungE{}
\def \KleidungF{}

% Ausrüstung
\def \AusruestungLinksA{}
\def \AusruestungLinksAGewicht{}
\def \AusruestungLinksAWoGetragen{}
\def \AusruestungRechtsA{}
\def \AusruestungRechtsAGewicht{}
\def \AusruestungRechtsAWoGetragen{}
\def \AusruestungLinksB{}
\def \AusruestungLinksBGewicht{}
\def \AusruestungLinksBWoGetragen{}
\def \AusruestungRechtsB{}
\def \AusruestungRechtsBGewicht{}
\def \AusruestungRechtsBWoGetragen{}
\def \AusruestungLinksC{}
\def \AusruestungLinksCGewicht{}
\def \AusruestungLinksCWoGetragen{}
\def \AusruestungRechtsC{}
\def \AusruestungRechtsCGewicht{}
\def \AusruestungRechtsCWoGetragen{}
\def \AusruestungLinksD{}
\def \AusruestungLinksDGewicht{}
\def \AusruestungLinksDWoGetragen{}
\def \AusruestungRechtsD{}
\def \AusruestungRechtsDGewicht{}
\def \AusruestungRechtsDWoGetragen{}
\def \AusruestungLinksE{}
\def \AusruestungLinksEGewicht{}
\def \AusruestungLinksEWoGetragen{}
\def \AusruestungRechtsE{}
\def \AusruestungRechtsEGewicht{}
\def \AusruestungRechtsEWoGetragen{}
\def \AusruestungLinksF{}
\def \AusruestungLinksFGewicht{}
\def \AusruestungLinksFWoGetragen{}
\def \AusruestungRechtsF{}
\def \AusruestungRechtsFGewicht{}
\def \AusruestungRechtsFWoGetragen{}
\def \AusruestungLinksG{}
\def \AusruestungLinksGGewicht{}
\def \AusruestungLinksGWoGetragen{}
\def \AusruestungRechtsG{}
\def \AusruestungRechtsGGewicht{}
\def \AusruestungRechtsGWoGetragen{}
\def \AusruestungLinksH{}
\def \AusruestungLinksHGewicht{}
\def \AusruestungLinksHWoGetragen{}
\def \AusruestungRechtsH{}
\def \AusruestungRechtsHGewicht{}
\def \AusruestungRechtsHWoGetragen{}
\def \AusruestungLinksI{}
\def \AusruestungLinksIGewicht{}
\def \AusruestungLinksIWoGetragen{}
\def \AusruestungRechtsI{}
\def \AusruestungRechtsIGewicht{}
\def \AusruestungRechtsIWoGetragen{}
\def \AusruestungLinksJ{}
\def \AusruestungLinksJGewicht{}
\def \AusruestungLinksJWoGetragen{}
\def \AusruestungRechtsJ{}
\def \AusruestungRechtsJGewicht{}
\def \AusruestungRechtsJWoGetragen{}
\def \AusruestungLinksK{}
\def \AusruestungLinksKGewicht{}
\def \AusruestungLinksKWoGetragen{}
\def \AusruestungRechtsK{}
\def \AusruestungRechtsKGewicht{}
\def \AusruestungRechtsKWoGetragen{}
\def \AusruestungLinksL{}
\def \AusruestungLinksLGewicht{}
\def \AusruestungLinksLWoGetragen{}
\def \AusruestungRechtsL{}
\def \AusruestungRechtsLGewicht{}
\def \AusruestungRechtsLWoGetragen{}
\def \AusruestungLinksM{}
\def \AusruestungLinksMGewicht{}
\def \AusruestungLinksMWoGetragen{}
\def \AusruestungRechtsM{}
\def \AusruestungRechtsMGewicht{}
\def \AusruestungRechtsMWoGetragen{}
\def \AusruestungLinksN{}
\def \AusruestungLinksNGewicht{}
\def \AusruestungLinksNWoGetragen{}
\def \AusruestungRechtsN{}
\def \AusruestungRechtsNGewicht{}
\def \AusruestungRechtsNWoGetragen{}
\def \AusruestungLinksO{}
\def \AusruestungLinksOGewicht{}
\def \AusruestungLinksOWoGetragen{}
\def \AusruestungRechtsO{}
\def \AusruestungRechtsOGewicht{}
\def \AusruestungRechtsOWoGetragen{}
\def \AusruestungLinksP{}
\def \AusruestungLinksPGewicht{}
\def \AusruestungLinksPWoGetragen{}
\def \AusruestungRechtsP{}
\def \AusruestungRechtsPGewicht{}
\def \AusruestungRechtsPWoGetragen{}
\def \AusruestungLinksQ{}
\def \AusruestungLinksQGewicht{}
\def \AusruestungLinksQWoGetragen{}
\def \AusruestungRechtsQ{}
\def \AusruestungRechtsQGewicht{}
\def \AusruestungRechtsQWoGetragen{}
\def \AusruestungLinksR{}
\def \AusruestungLinksRGewicht{}
\def \AusruestungLinksRWoGetragen{}
\def \AusruestungRechtsR{}
\def \AusruestungRechtsRGewicht{}
\def \AusruestungRechtsRWoGetragen{}

% Proviant/Tränke
\def \ProviantLinksA{}
\def \ProviantLinksARationen{}
\def \ProviantRechtsA{}
\def \ProviantRechtsARationen{}
\def \ProviantLinksB{}
\def \ProviantLinksBRationen{}
\def \ProviantRechtsB{}
\def \ProviantRechtsBRationen{}
\def \ProviantLinksC{}
\def \ProviantLinksCRationen{}
\def \ProviantRechtsC{}
\def \ProviantRechtsCRationen{}
\def \ProviantLinksD{}
\def \ProviantLinksDRationen{}
\def \ProviantRechtsD{}
\def \ProviantRechtsDRationen{}

% Vermögen
\def \VermoegenDukaten{0}
\def \VermoegenSilbertaler{0}
\def \VermoegenHeller{0}
\def \VermoegenKreuzer{0}

% Notizen
\def \NotizA{}
\def \NotizB{}
\def \NotizC{}
\def \NotizD{$\parallel$ Generiert: \CharakterNotizGenerationsdatum\ $\parallel$ Letzte Änderung: \CharakterNotizLetzteAenderung\ $\parallel$ Kompiliert: \CharakterNotizKompilierungsDatum\ $\parallel$ Erlebte Abenteuer: \CharakterNotizAnzahlAbenteuer\ $\parallel$}
 % Charakter Konfiguration laden

\begin{document}
%----------------------------
%	The fancy stuff
% dieses Makro gibt das Zeichen für eine Sonderfertigkeit zurück
% Also angekreuzt wenn gewählt und leer wenn nicht
\def \dasKreuzchen{$\boxtimes$}%<-- das Kreuzchen Symbol
\def \dasNichtKreuzchen{$\square$}%<-- das NichtKreuzchen Symbol
\newcommand{\SFKreuzchen}[1]{%
\IfSubStr{\SonderfertigkeitenNahkampfA\SonderfertigkeitenNahkampfB\SonderfertigkeitenNahkampfC
\SFausserKampfA\SFausserKampfB\SFausserKampfC\SFausserKampfD\SFausserKampfE\SFausserKampfF}{#1}
{\dasKreuzchen}{\dasNichtKreuzchen}%
}

%----------------------------
%	Kampfwerte
\def \KampfwerteTrenner{\ $\bullet$\ }
% Basis Kampftechniken
\def \KampfwerteDolche{\makebox[2.5mm][c]{\TalentDolcheAT}\KampfwerteTrenner\makebox[2.5mm][c]{\TalentDolchePA}}
\def \KampfwerteHiebwaffen{\makebox[2.5mm][c]{\TalentHiebwaffenAT}\KampfwerteTrenner\makebox[2.5mm][c]{\TalentHiebwaffenPA}}
\def \KampfwerteRaufen{\makebox[2.5mm][c]{\TalentRaufenAT}\KampfwerteTrenner\makebox[2.5mm][c]{\TalentRaufenPA}}
\def \KampfwerteRingen{\makebox[2.5mm][c]{\TalentRingenAT}\KampfwerteTrenner\makebox[2.5mm][c]{\TalentRingenPA}}
\def \KampfwerteSaebel{\makebox[2.5mm][c]{\TalentSaebelAT}\KampfwerteTrenner\makebox[2.5mm][c]{\TalentSaebelPA}}
\def \KampfwerteWurfmesser{\makebox[2.5mm][c]{\TalentWurfmesserAT}\KampfwerteTrenner\makebox[2.5mm][c]{-}}

% Extra Kampftechniken
\def \KampfwerteKampfExtraA{\makebox[2.5mm][c]{\TalentKampfExtraAAT}\KampfwerteTrenner\makebox[2.5mm][c]{\TalentKampfExtraAPA}}
\def \KampfwerteKampfExtraB{\makebox[2.5mm][c]{\TalentKampfExtraBAT}\KampfwerteTrenner\makebox[2.5mm][c]{\TalentKampfExtraBPA}}
\def \KampfwerteKampfExtraC{\makebox[2.5mm][c]{\TalentKampfExtraCAT}\KampfwerteTrenner\makebox[2.5mm][c]{\TalentKampfExtraCPA}}
\def \KampfwerteKampfExtraD{\makebox[2.5mm][c]{\TalentKampfExtraDAT}\KampfwerteTrenner\makebox[2.5mm][c]{\TalentKampfExtraDPA}}
\def \KampfwerteKampfExtraE{\makebox[2.5mm][c]{\TalentKampfExtraEAT}\KampfwerteTrenner\makebox[2.5mm][c]{\TalentKampfExtraEPA}}
\def \KampfwerteKampfExtraF{\makebox[2.5mm][c]{\TalentKampfExtraFAT}\KampfwerteTrenner\makebox[2.5mm][c]{\TalentKampfExtraFPA}}
\def \KampfwerteKampfExtraG{\makebox[2.5mm][c]{\TalentKampfExtraGAT}\KampfwerteTrenner\makebox[2.5mm][c]{\TalentKampfExtraGPA}}
\def \KampfwerteKampfExtraH{\makebox[2.5mm][c]{\TalentKampfExtraHAT}\KampfwerteTrenner\makebox[2.5mm][c]{\TalentKampfExtraHPA}}
\def \KampfwerteKampfExtraI{\makebox[2.5mm][c]{\TalentKampfExtraIAT}\KampfwerteTrenner\makebox[2.5mm][c]{\TalentKampfExtraIPA}}
\def \KampfwerteKampfExtraJ{\makebox[2.5mm][c]{\TalentKampfExtraJAT}\KampfwerteTrenner\makebox[2.5mm][c]{\TalentKampfExtraJPA}}

%----------------------------
%	Zauberbefüllung
\makeatletter
\newcounter{ZauberCounter}
\ifnum\value{ZauberAnzahl}>32
	\forloop[1]{ZauberCounter}{\value{ZauberAnzahl}}{\value{ZauberCounter} < 66}{%
		\g@addto@macro\ZauberTabelleInhaltB{&&&&&&&&&&&&&\\\hline}%
	}
\else
	\forloop{ZauberCounter}{\value{ZauberAnzahl}}{\value{ZauberCounter} < 33}{%
		\g@addto@macro\ZauberTabelleInhaltA{&&&&&&&&&&&&&\\\hline}%
	}
\fi
\makeatother % Automatische Makros laden
\renewcommand{\arraystretch}{1.4} % set the space between rows:
%
% DSA Logo & Überschrift
\begin{center}
\includegraphics[width=63mm]{DSALogo.png}\\[3mm]
{\Huge \textbf{Heldendokument}}
\end{center}
%
% 1. Reihe
\begin{tabular}{!{\VRule[3pt]}lp{18mm}p{59mm}p{31mm}p{53mm}l!{\VRule[3pt]}}
\specialrule{3pt}{0pt}{0pt}
&\textbf{Name} & \CharakterName & \textbf{GP Basis} & \CharakterGPBasis &\\\cline{2-5}
&\textbf{Rasse} & \CharakterRasse & \textbf{(Modifikationen):} & \CharakterRasseModifikationen &\\\cline{2-5}
&\textbf{Kultur} & \CharakterKultur & \textbf{(Modifikationen):} & \CharakterKulturModifikationen  &\\\cline{2-5}
&\textbf{Profession} & \CharakterProfession & \textbf{(Modifikationen):} & \CharakterProfessionModifikationen  &\\\specialrule{3pt}{0pt}{0pt}
\end{tabular}
\\[5mm]
%
% 2. Reihe
\begin{tabular}{!{\VRule[3pt]}lp{20.7mm}p{32.3mm}l!{\VRule[3pt]}}
\specialrule{3pt}{0pt}{0pt}
&\textbf{Geschlecht} & \CharakterGeschlecht &\\\cline{2-3}
&\textbf{Alter} & \CharakterAlter &\\\cline{2-3}
&\textbf{Grösse} & \CharakterGroesse &\\\cline{2-3}
&\textbf{Gewicht} & \CharakterGewicht &\\\cline{2-3}
&\textbf{Haarfarbe} & \CharakterHaarfarbe &\\\cline{2-3}
&\textbf{Augenfarbe} & \CharakterAugenfarbe &\\\cline{2-3}
&\textbf{Aussehen} & \CharakterAussehenA &\\\cline{2-3}
& \multicolumn{2}{l}{\CharakterAussehenB} &\\\cline{2-3}
& \multicolumn{2}{l}{\CharakterAussehenC} &\\\cline{2-3}
& \multicolumn{2}{l}{\CharakterAussehenD} &\\\cline{2-3}
\specialrule{3pt}{0pt}{0pt}
\end{tabular}
\begin{tabular}{!{\VRule[3pt]}p{36mm}!{\VRule[3pt]}}
\specialrule{3pt}{0pt}{0pt}
Wappen/Portrait\\
\includegraphics[width=36mm]{\CharakterPortraitBild}\\
\specialrule{3pt}{0pt}{0pt}
\end{tabular}
\begin{tabular}{!{\VRule[3pt]}lp{20.7mm}p{32.3mm}l!{\VRule[3pt]}}
\specialrule{3pt}{0pt}{0pt}
&\textbf{Stand} & \CharakterStand &\\\cline{2-3}
&\textbf{Titel} & \CharakterTitel &\\\cline{2-3}
&\textbf{Sozialstatus} & \CharakterSO &\\\cline{2-3}
& \multicolumn{2}{l}{\textbf{Familie/Herkunft/Hintergrund}}  &\\\cline{2-3}
& \multicolumn{2}{l}{\CharakterHintergrundA} &\\\cline{2-3}
& \multicolumn{2}{l}{\CharakterHintergrundB} &\\\cline{2-3}
& \multicolumn{2}{l}{\CharakterHintergrundC} &\\\cline{2-3}
& \multicolumn{2}{l}{\CharakterHintergrundD} &\\\cline{2-3}
& \multicolumn{2}{l}{\CharakterHintergrundE} &\\\cline{2-3}
& \multicolumn{2}{l}{\CharakterHintergrundF} &\\\cline{2-3}
\specialrule{3pt}{0pt}{0pt}
\end{tabular}
\vspace*{5mm}
%
% Vorteile
\newline
\begin{tabular}{>{\centering}m{89mm} >{\centering}m{89mm}}
\Huge \textbf{Vorteile} & \Huge \textbf{Nachteile}
\end{tabular}
\vspace*{2mm}
\newline
\begin{tabular}{!{\VRule[3pt]}lp{79mm}l!{\VRule[3pt]}}
\specialrule{3pt}{0pt}{0pt}
& \CharakterVorteileA &\\\cline{2-2}
& \CharakterVorteileB &\\\cline{2-2}
& \CharakterVorteileC &\\\cline{2-2}
& \CharakterVorteileD &\\\cline{2-2}
& \CharakterVorteileE &\\
\specialrule{3pt}{0pt}{0pt}
\end{tabular}
% Nachteile
\begin{tabular}{!{\VRule[3pt]}lp{79mm}l!{\VRule[3pt]}}
\specialrule{3pt}{0pt}{0pt}
& \CharakterNachteileA &\\\cline{2-2}
& \CharakterNachteileB &\\\cline{2-2}
& \CharakterNachteileC &\\\cline{2-2}
& \CharakterNachteileD &\\\cline{2-2}
& \CharakterNachteileE &\\
\specialrule{3pt}{0pt}{0pt}
\end{tabular}
\vspace*{3mm}
%
% Eigenschaften & Basiswerte
\begin{center}
{\Huge \textbf{Eigenschaften \& Basiswerte}}\\[3mm]
\end{center}
\begin{tabular}{
		!{\VRule[3pt]}p{29mm}
		|>{\centering\arraybackslash}p{7.5mm}
		|>{\centering\arraybackslash}p{7.5mm}
		!{\VRule[2pt]}>{\centering\arraybackslash}p{7.5mm}
		!{\VRule[3pt]}
	}
\specialrule{3pt}{0pt}{0pt}
& {\tiny Mod.} & {\tiny Start} & {\tiny Aktuel}\\\hline
\textbf{Mut} & \EigMUmod & \EigMUstart & \EigMUaktuell \\\hline
\textbf{Klugheit} & \EigKLmod & \EigKLstart & \EigKLaktuell \\\hline
\textbf{Intuition} & \EigINmod & \EigINstart & \EigINaktuell \\\hline
\textbf{Charisma} & \EigCHmod & \EigCHstart & \EigCHaktuell \\\hline
\textbf{Fingerfertigkeit} & \EigFFmod & \EigFFstart & \EigFFaktuell \\\hline
\textbf{Gewandtheit} & \EigGEmod & \EigGEstart & \EigGEaktuell \\\hline
\textbf{Konstitution} & \EigKOmod & \EigKOstart & \EigKOaktuell \\\hline
\textbf{Körperkraft} & \EigKKmod & \EigKKstart & \EigKKaktuell \\\hline
\textbf{Geschwindigkeit} & \EigGSmod & \EigGSstart & \EigGSaktuell \\
\specialrule{3pt}{0pt}{0pt}
\end{tabular}
\begin{tabular}{
		!{\VRule[3pt]}p{46mm}
		|>{\centering\arraybackslash}p{7mm}
		|>{\centering\arraybackslash}p{7mm}
		!{\VRule[2pt]}>{\centering\arraybackslash}p{7mm}
		!{\VRule[2pt]}>{\centering\arraybackslash}p{9.5mm}
		|>{\centering\arraybackslash}p{9.5mm}
		!{\VRule[3pt]}
	}
\specialrule{3pt}{0pt}{0pt}
& {\tiny Mod.} & {\tiny Start} & {\tiny Aktuel} & {\tiny Zugek.} & {\tiny max Z}\\\hline
\textbf{Lebenspunkte} & \BasisLEmod & \BasisLEstart & \BasisLEaktuell & \BasisLEzugekauft & \BasisLEmaxZugekauft \\\hline
\textbf{Ausdauer} & \BasisAUmod & \BasisAUstart & \BasisAUaktuell & \BasisAUzugekauft & \BasisAUmaxZugekauft \\\hline
\textbf{Astralenergie} & \BasisAEmod & \BasisAEstart & \BasisAEaktuell & \BasisAEzugekauft & \BasisAEmaxZugekauft \\\hline
\textbf{Karmaenergie} & \BasisKEmod & \BasisKEstart & \BasisKEaktuell & \BasisKEzugekauft & \BasisKEmaxZugekauft \\\hline
\textbf{Magieresistenz} & \BasisMRmod & \BasisMRstart & \BasisMRaktuell & \BasisMRzugekauft & \BasisMRmaxZugekauft \\\hline
\textbf{INI-Basiswert} & \BasisINImod & \BasisINIstart & \BasisINIaktuell & \multicolumn{2}{p{19mm}!{\VRule[3pt]}}{\vspace{-15pt}\tiny \SFKreuzchen{Kampfgespür}Kampfgespür(+2) \SFKreuzchen{Kampfreflexe}Kampfreflexe(+4)\vspace{-2pt}}\\\hline
\textbf{AT-Basiswert} & \BasisATmod & \BasisATstart & \BasisATaktuell &  \multicolumn{2}{p{19mm}!{\VRule[3pt]}}{\multirow{3}{19mm}{{\footnotesize Wundschwelle} {\tiny (KO/2), Mod} \large \BasisWundschwelle}}\\\cline{1-4}
\textbf{PA-Basiswert} & \BasisPAmod & \BasisPAstart & \BasisPAaktuell & \multicolumn{2}{p{19mm}!{\VRule[3pt]}}{}\\\cline{1-4}
\textbf{FK-Basiswert} & \BasisFKmod & \BasisFKstart & \BasisFKaktuell & \multicolumn{2}{p{19mm}!{\VRule[3pt]}}{}\\\cline{1-4}
\specialrule{3pt}{0pt}{0pt}
\end{tabular}
\vspace*{3mm}
%
% Abenteuerpunkte
\begin{center}
{\Huge \textbf{Abenteuerpunkte}}\\[3mm]
\end{center}
\begin{tabular}{!{\VRule[3pt]}p{29mm}p{24.9mm}p{29mm}p{24.9mm}p{29mm}p{24.9mm}!{\VRule[3pt]}}
\specialrule{3pt}{0pt}{0pt}
\textbf{Gesamt:} & \APgesamt & \textbf{Eingesetzte AP:} & \APeingesetzt & \textbf{Guthaben:} & \APguthaben \\
\specialrule{3pt}{0pt}{0pt}
\end{tabular}
\vfill
{\footnotesize \footline}
 % erste Seite (Übersicht, Eigenschaften & Basiswerte)
\newpage
%
% Eigenschaftsübersicht
\renewcommand{\arraystretch}{1.05}
\vspace*{3mm}
\hspace*{-6.3mm}
{\Large
\begin{tabular}{!{\VRule[3pt]}p{17.45mm}p{16.45mm}p{16.45mm}p{16.45mm}p{16.45mm}p{16.45mm}p{16.45mm}p{17mm}!{\VRule[3pt]}p{16mm}!{\VRule[3pt]}}
\specialrule{3pt}{0pt}{0pt}
\textbf{MU:} \EigMUaktuell & \textbf{KL:} \EigKLaktuell & \textbf{IN:} \EigINaktuell & \textbf{CH:} \EigCHaktuell & \textbf{FF:} \EigFFaktuell & \textbf{GE:} \EigGEaktuell & \textbf{KO:} \EigKOaktuell & \textbf{KK:} \EigKKaktuell & \textbf{BE:} \RuestungSummeBE\\
\specialrule{3pt}{0pt}{0pt}
\end{tabular}
}
\vspace*{2mm}
%
% Sonderfertigkeiten, Gaben & Talente
\begin{center}
{\Huge \textbf{Sonderfertigkeiten, Gaben \& Talente}}\\[2mm]
\end{center}
{\small
\begin{tabular}{
		!{\VRule[3pt]}l
		|p{32mm}
		|p{3mm}
		|p{8mm}
		|c
		|p{3.5mm}
		|p{3.5mm}
		!{\VRule[3pt]}l
		|p{55mm}
		|p{10mm}
		|p{3.5mm}
		|p{3.5mm}
		!{\VRule[3pt]}
	}
\specialrule{3pt}{0pt}{0pt}
% Sonderfertigkeiten (Ausser Kampf) | Natur-Talente
\multicolumn{7}{!{\VRule[3pt]}c!{\VRule[3pt]}}{\multirow{2}{*}{\Large \textbf{Sonderfertigkeiten (Ausser Kampf)}}} & \multicolumn{3}{l|}{\multirow{2}{*}{\Large \textbf{Natur-Talente (B)}}} & \multicolumn{2}{c!{\VRule[3pt]}}{\multirow{2}{*}{\textbf{TaW}}}\\
\multicolumn{7}{!{\VRule[3pt]}c!{\VRule[3pt]}}{} & \multicolumn{3}{l|}{} & \multicolumn{2}{c!{\VRule[3pt]}}{}\\\hline
\multicolumn{7}{!{\VRule[3pt]}l!{\VRule[3pt]}}{\SFausserKampfA}&&\multicolumn{2}{l|}{Fährtensuchen (KL $\bullet$ IN $\bullet$ IN/KO)} & \TalentFaehrtensuchenTaWL & \TalentFaehrtensuchenTaWR\\\hline
\multicolumn{7}{!{\VRule[3pt]}l!{\VRule[3pt]}}{\SFausserKampfB}&&\multicolumn{2}{l|}{Orientierung (KL $\bullet$ IN $\bullet$ IN)} & \TalentOrientierungTaWL & \TalentOrientierungTaWR\\\hline
\multicolumn{7}{!{\VRule[3pt]}l!{\VRule[3pt]}}{\SFausserKampfC}&&\multicolumn{2}{l|}{Wildnisleben (IN $\bullet$ GE $\bullet$ KO)} & \TalentWildnislebenTaWL & \TalentWildnislebenTaWR \\\hline
\multicolumn{7}{!{\VRule[3pt]}l!{\VRule[3pt]}}{\SFausserKampfD}&&\multicolumn{2}{l|}{\TalentNaturExtraA\ \TalentNaturExtraAProbe} & \TalentNaturExtraATaWL & \TalentNaturExtraATaWR\\\hline
\multicolumn{7}{!{\VRule[3pt]}l!{\VRule[3pt]}}{\SFausserKampfE}&&\multicolumn{2}{l|}{\TalentNaturExtraB\ \TalentNaturExtraBProbe} & \TalentNaturExtraBTaWL & \TalentNaturExtraBTaWR\\\hline
\multicolumn{7}{!{\VRule[3pt]}l!{\VRule[3pt]}}{\SFausserKampfF}&&\multicolumn{2}{l|}{\TalentNaturExtraC\ \TalentNaturExtraCProbe} & \TalentNaturExtraCTaWL & \TalentNaturExtraCTaWR\\\hline
\multicolumn{7}{!{\VRule[3pt]}l!{\VRule[3pt]}}{}&&\multicolumn{2}{l|}{\TalentNaturExtraD\ \TalentNaturExtraDProbe} & \TalentNaturExtraDTaWL & \TalentNaturExtraDTaWR\\\Cline{2pt}{1-7}\cline{8-12}
% Gaben | Wissenstalente
\multicolumn{5}{!{\VRule[3pt]}l|}{\multirow{2}{*}{\Large \textbf{Gaben (F)}}} & \multicolumn{2}{c!{\VRule[3pt]}}{\multirow{2}{*}{\textbf{TaW}}} & \multicolumn{3}{l|}{\multirow{2}{*}{\Large \textbf{Wissens-Talente (B)}}} & \multicolumn{2}{c!{\VRule[3pt]}}{\multirow{2}{*}{\textbf{TaW}}}\\
\multicolumn{5}{!{\VRule[3pt]}l|}{} & \multicolumn{2}{c!{\VRule[3pt]}}{} & \multicolumn{3}{l|}{} & \multicolumn{2}{c!{\VRule[3pt]}}{}\\\hline
& \multicolumn{4}{l|}{\GabeA} & \GabeATaWL & \GabeATaWR & &\multicolumn{2}{l|}{Götter/Kulte (KL $\bullet$ KL $\bullet$ IN)} & \TalentGoetterKulteTaWL & \TalentGoetterKulteTaWR \\\hline
& \multicolumn{4}{l|}{\GabeB} & \GabeBTaWL & \GabeBTaWR & &\multicolumn{2}{l|}{Rechnen (KL $\bullet$ KL $\bullet$ IN)} & \TalentRechnenTaWL & \TalentRechnenTaWR \\\hline
& \multicolumn{4}{l|}{\GabeC} & \GabeCTaWL & \GabeCTaWR & &\multicolumn{2}{l|}{Sagen/Legenden (KL $\bullet$ IN $\bullet$ CH)} & \TalentSagenLegendenTaWL & \TalentSagenLegendenTaWR \\\Cline{2pt}{1-7}\cline{8-12}
% Kampftechniken | Wissenstalente
\multicolumn{3}{!{\VRule[3pt]}l|}{\multirow{2}{*}{\Large \textbf{Kampftechniken}}} & \multirow{2}{*}{\textbf{BE}} & \multirow{2}{*}{\textbf{AT}$\bullet$\textbf{PA}} & \multicolumn{2}{c!{\VRule[3pt]}}{\multirow{2}{*}{\textbf{TaW}}} &&\multicolumn{2}{l|}{\TalentWissenExtraA\ \TalentWissenExtraAProbe} & \TalentWissenExtraATaWL & \TalentWissenExtraATaWR \\\cline{8-12}
\multicolumn{3}{!{\VRule[3pt]}l|}{} & \multirow{2}{*}{} & \multirow{2}{*}{} & \multicolumn{2}{c!{\VRule[3pt]}}{} &&\multicolumn{2}{l|}{\TalentWissenExtraB\ \TalentWissenExtraBProbe} & \TalentWissenExtraBTaWL & \TalentWissenExtraBTaWR \\\hline
& Dolche     & D & BE-1 & \KampfwerteDolche & \TalentDolcheTaWL & \TalentDolcheTaWR &&\multicolumn{2}{l|}{\TalentWissenExtraC\ \TalentWissenExtraCProbe} & \TalentWissenExtraCTaWL & \TalentWissenExtraCTaWR \\\hline
& Hiebwaffen & D & BE-4 & \KampfwerteHiebwaffen & \TalentHiebwaffenTaWL & \TalentHiebwaffenTaWR &&\multicolumn{2}{l|}{\TalentWissenExtraD\ \TalentWissenExtraDProbe} & \TalentWissenExtraDTaWL & \TalentWissenExtraDTaWR \\\hline
& Raufen     & C & BE   & \KampfwerteRaufen & \TalentRaufenTaWL & \TalentRaufenTaWR &&\multicolumn{2}{l|}{\TalentWissenExtraE\ \TalentWissenExtraEProbe} & \TalentWissenExtraETaWL & \TalentWissenExtraETaWR \\\hline
& Ringen     & D & BE   & \KampfwerteRingen & \TalentRingenTaWL & \TalentRingenTaWR &&\multicolumn{2}{l|}{\TalentWissenExtraF\ \TalentWissenExtraFProbe} & \TalentWissenExtraFTaWL & \TalentWissenExtraFTaWR \\\hline
& Säbel      & D & BE-2 & \KampfwerteSaebel & \TalentSaebelTaWL & \TalentSaebelTaWR &&\multicolumn{2}{l|}{\TalentWissenExtraG\ \TalentWissenExtraGProbe} & \TalentWissenExtraGTaWL & \TalentWissenExtraGTaWR \\\hline
& Wurfmesser & C & BE-3 & \KampfwerteWurfmesser & \TalentWurfmesserTaWL & \TalentWurfmesserTaWR &&\multicolumn{2}{l|}{\TalentWissenExtraH\ \TalentWissenExtraHProbe} & \TalentWissenExtraHTaWL & \TalentWissenExtraHTaWR \\\hline
& \TalentKampfExtraA & \TalentKampfExtraASKat & \TalentKampfExtraABE & \KampfwerteKampfExtraA & \TalentKampfExtraATaWL & \TalentKampfExtraATaWR &&\multicolumn{2}{l|}{\TalentWissenExtraI\ \TalentWissenExtraIProbe} & \TalentWissenExtraITaWL & \TalentWissenExtraITaWR \\\hline
& \TalentKampfExtraB & \TalentKampfExtraBSKat & \TalentKampfExtraBBE & \KampfwerteKampfExtraB & \TalentKampfExtraBTaWL & \TalentKampfExtraBTaWR &&\multicolumn{2}{l|}{\TalentWissenExtraJ\ \TalentWissenExtraJProbe} & \TalentWissenExtraJTaWL & \TalentWissenExtraJTaWR \\\hline
& \TalentKampfExtraC & \TalentKampfExtraCSKat & \TalentKampfExtraCBE & \KampfwerteKampfExtraC & \TalentKampfExtraCTaWL & \TalentKampfExtraCTaWR &&\multicolumn{2}{l|}{\TalentWissenExtraK\ \TalentWissenExtraKProbe} & \TalentWissenExtraKTaWL & \TalentWissenExtraKTaWR \\\hline
& \TalentKampfExtraD & \TalentKampfExtraDSKat & \TalentKampfExtraDBE & \KampfwerteKampfExtraD & \TalentKampfExtraDTaWL & \TalentKampfExtraDTaWR &&\multicolumn{2}{l|}{\TalentWissenExtraL\ \TalentWissenExtraLProbe} & \TalentWissenExtraLTaWL & \TalentWissenExtraLTaWR \\\hline
& \TalentKampfExtraE & \TalentKampfExtraESKat & \TalentKampfExtraEBE & \KampfwerteKampfExtraE & \TalentKampfExtraETaWL & \TalentKampfExtraETaWR &&\multicolumn{2}{l|}{\TalentWissenExtraM\ \TalentWissenExtraMProbe} & \TalentWissenExtraMTaWL & \TalentWissenExtraMTaWR \\\hline
& \TalentKampfExtraF & \TalentKampfExtraFSKat & \TalentKampfExtraFBE & \KampfwerteKampfExtraF & \TalentKampfExtraFTaWL & \TalentKampfExtraFTaWR &&\multicolumn{2}{l|}{\TalentWissenExtraN\ \TalentWissenExtraNProbe} & \TalentWissenExtraNTaWL & \TalentWissenExtraNTaWR \\\hline
% Kampftechniken | Sprachen & Schriften
& \TalentKampfExtraG & \TalentKampfExtraGSKat & \TalentKampfExtraGBE & \KampfwerteKampfExtraG & \TalentKampfExtraGTaWL & \TalentKampfExtraGTaWR & \multicolumn{2}{l|}{\multirow{2}{*}{\Large \textbf{Sprachen \& Schriften}}} & \multirow{2}{*}{\textbf{Komp}} & \multicolumn{2}{c!{\VRule[3pt]}}{\multirow{2}{*}{\textbf{TaW}}}\\\cline{1-7}
& \TalentKampfExtraH & \TalentKampfExtraHSKat & \TalentKampfExtraHBE & \KampfwerteKampfExtraH & \TalentKampfExtraHTaWL & \TalentKampfExtraHTaWR & \multicolumn{2}{l|}{} & & \multicolumn{2}{c!{\VRule[3pt]}}{}\\\hline
& \TalentKampfExtraI & \TalentKampfExtraISKat & \TalentKampfExtraIBE & \KampfwerteKampfExtraI & \TalentKampfExtraITaWL & \TalentKampfExtraITaWR && Muttersprache: \TalentMuttersprache & \TalentMutterspracheKomp & \TalentMutterspracheTaWL & \TalentMutterspracheTaWR \\\hline
& \TalentKampfExtraJ & \TalentKampfExtraJSKat & \TalentKampfExtraJBE & \KampfwerteKampfExtraJ & \TalentKampfExtraJTaWL & \TalentKampfExtraJTaWR && \TalentSpracheSchriftExtraA & \TalentSpracheSchriftExtraAKomp & \TalentSpracheSchriftExtraATaWL & \TalentSpracheSchriftExtraATaWR \\\hline
% Körperliche Talente | Sprachen & Schriften
\multicolumn{5}{!{\VRule[3pt]}l|}{\multirow{2}{*}{\Large \textbf{Körperliche Talente}}}  & \multicolumn{2}{c!{\VRule[3pt]}}{\multirow{2}{*}{\textbf{TaW}}} && \TalentSpracheSchriftExtraB & \TalentSpracheSchriftExtraBKomp & \TalentSpracheSchriftExtraBTaWL & \TalentSpracheSchriftExtraBTaWR\\\cline{8-12}
\multicolumn{5}{!{\VRule[3pt]}l|}{} & \multicolumn{2}{c!{\VRule[3pt]}}{} && \TalentSpracheSchriftExtraC & \TalentSpracheSchriftExtraCKomp & \TalentSpracheSchriftExtraCTaWL & \TalentSpracheSchriftExtraCTaWR \\\hline
& \multicolumn{3}{l|}{Athletik (GE $\bullet$ KO $\bullet$ KK)}           & BEx2 & \TalentAthletikTaWL            & \TalentAthletikTaWR                && \TalentSpracheSchriftExtraD & \TalentSpracheSchriftExtraDKomp & \TalentSpracheSchriftExtraDTaWL & \TalentSpracheSchriftExtraDTaWR \\\hline
& \multicolumn{3}{l|}{Klettern (MU $\bullet$ GE $\bullet$ KK)}           & BEx2 & \TalentKletternTaWL            & \TalentKletternTaWR                && \TalentSpracheSchriftExtraE & \TalentSpracheSchriftExtraEKomp & \TalentSpracheSchriftExtraETaWL & \TalentSpracheSchriftExtraETaWR \\\hline
& \multicolumn{3}{l|}{Körperbeherrschung (MU$\bullet$IN$\bullet$GE)} & BEx2 & \TalentKoerperbeherrschungTaWL & \TalentKoerperbeherrschungTaWR     && \TalentSpracheSchriftExtraF & \TalentSpracheSchriftExtraFKomp & \TalentSpracheSchriftExtraFTaWL & \TalentSpracheSchriftExtraFTaWR \\\hline
& \multicolumn{3}{l|}{Schleichen (MU $\bullet$ IN $\bullet$ GE)}         &  BE  & \TalentSchleichenTaWL          & \TalentSchleichenTaWR              && \TalentSpracheSchriftExtraG & \TalentSpracheSchriftExtraGKomp & \TalentSpracheSchriftExtraGTaWL & \TalentSpracheSchriftExtraGTaWR \\\hline
& \multicolumn{3}{l|}{Schwimmen (GE $\bullet$ KO $\bullet$ KK)}          & BEx2 & \TalentSchwimmenTaWL           & \TalentSchwimmenTaWR               && \TalentSpracheSchriftExtraH & \TalentSpracheSchriftExtraHKomp & \TalentSpracheSchriftExtraHTaWL & \TalentSpracheSchriftExtraHTaWR \\\hline
& \multicolumn{3}{l|}{Selbsbeherrschung (MU$\bullet$KO$\bullet$KK)}  & ---  & \TalentSelbstbeherrschungTaWL  & \TalentSelbstbeherrschungTaWR      && \TalentSpracheSchriftExtraI & \TalentSpracheSchriftExtraIKomp & \TalentSpracheSchriftExtraITaWL & \TalentSpracheSchriftExtraITaWR \\\hline
& \multicolumn{3}{l|}{Sich Verstecken (MU $\bullet$ IN $\bullet$ GE)}    & BE-2 & \TalentSichVersteckenTaWL      & \TalentSichVersteckenTaWR          && \TalentSpracheSchriftExtraJ & \TalentSpracheSchriftExtraJKomp & \TalentSpracheSchriftExtraJTaWL & \TalentSpracheSchriftExtraJTaWR \\\hline
% Körperliche Talente | Handwerkliche Talente
& \multicolumn{3}{l|}{Singen (IN $\bullet$ CH $\bullet$ CH/KO)}          & BE-3 & \TalentSingenTaWL              & \TalentSingenTaWR                  &\multicolumn{3}{l|}{\multirow{2}{*}{\Large \textbf{Handwerkliche Talente (B)}}} & \multicolumn{2}{c!{\VRule[3pt]}}{\multirow{2}{*}{\textbf{TaW}}}\\\cline{1-7}
& \multicolumn{3}{l|}{Sinnesschärfe (KL $\bullet$ IN $\bullet$ IN/FF)}   & spez & \TalentSinnesschaerfeTaWL      & \TalentSinnesschaerfeTaWR          &\multicolumn{3}{l|}{} & \multicolumn{2}{c!{\VRule[3pt]}}{}\\\hline
& \multicolumn{3}{l|}{Tanzen (CH $\bullet$ GE $\bullet$ GE)}             & BEx2 & \TalentTanzenTaWL              & \TalentTanzenTaWR                  && \multicolumn{2}{l|}{Heilkunde Wunden (KL $\bullet$ CH $\bullet$ FF)}   & \TalentHeilkundeWundenTaWL & \TalentHeilkundeWundenTaWR \\\hline
& \multicolumn{3}{l|}{Zechen (IN $\bullet$ KO $\bullet$ KK)}             & ---  & \TalentZechenTaWL              & \TalentZechenTaWR                  && \multicolumn{2}{l|}{Holzbearbeitung (KL $\bullet$ FF $\bullet$ KK)}    & \TalentHolzbearbeitungTaWL & \TalentHolzbearbeitungTaWR\\\hline
& \multicolumn{3}{l|}{\TalentKoerperExtraA\ \TalentKoerperExtraAProbe} & \TalentKoerperExtraABE & \TalentKoerperExtraATaWL & \TalentKoerperExtraATaWR && \multicolumn{2}{l|}{Kochen (KL $\bullet$ IN $\bullet$ FF)}             & \TalentKochenTaWL          & \TalentKochenTaWR\\\hline
& \multicolumn{3}{l|}{\TalentKoerperExtraB\ \TalentKoerperExtraBProbe} & \TalentKoerperExtraBBE & \TalentKoerperExtraBTaWL & \TalentKoerperExtraBTaWR && \multicolumn{2}{l|}{Lederarbeiten (KL $\bullet$ FF $\bullet$ FF)}      & \TalentLederarbeitenTaWL   & \TalentLederarbeitenTaWR\\\hline
& \multicolumn{3}{l|}{\TalentKoerperExtraC\ \TalentKoerperExtraCProbe} & \TalentKoerperExtraCBE & \TalentKoerperExtraCTaWL & \TalentKoerperExtraCTaWR && \multicolumn{2}{l|}{Malen/Zeichnen (KL $\bullet$ IN $\bullet$ FF)}     & \TalentMalenZeichnenTaWL   & \TalentMalenZeichnenTaWR\\\hline
& \multicolumn{3}{l|}{\TalentKoerperExtraD\ \TalentKoerperExtraDProbe} & \TalentKoerperExtraDBE & \TalentKoerperExtraDTaWL & \TalentKoerperExtraDTaWR && \multicolumn{2}{l|}{Schneidern (KL $\bullet$ FF $\bullet$ FF)}         & \TalentSchneidernTaWL      & \TalentSchneidernTaWR\\\hline
& \multicolumn{3}{l|}{\TalentKoerperExtraE\ \TalentKoerperExtraEProbe} & \TalentKoerperExtraEBE & \TalentKoerperExtraETaWL & \TalentKoerperExtraETaWR && \multicolumn{2}{l|}{\TalentHandwerkExtraA\ \TalentHandwerkExtraAProbe} & \TalentHandwerkExtraATaWL  & \TalentHandwerkExtraATaWR\\\hline
% Gesellschaftliche Talente | Handwerkliche Talente
\multicolumn{5}{!{\VRule[3pt]}l|}{\multirow{2}{*}{\Large \textbf{Gesellschaftliche Talente}}}  & \multicolumn{2}{c!{\VRule[3pt]}}{\multirow{2}{*}{\textbf{TaW}}} & & \multicolumn{2}{l|}{\TalentHandwerkExtraB\ \TalentHandwerkExtraBProbe} & \TalentHandwerkExtraBTaWL  & \TalentHandwerkExtraBTaWR\\\cline{8-12}
\multicolumn{5}{!{\VRule[3pt]}l|}{} & \multicolumn{2}{c!{\VRule[3pt]}}{} & & \multicolumn{2}{l|}{\TalentHandwerkExtraC\ \TalentHandwerkExtraCProbe} & \TalentHandwerkExtraCTaWL  & \TalentHandwerkExtraCTaWR\\\hline
& \multicolumn{4}{l|}{Menschenkenntnis (KL $\bullet$ IN $\bullet$ CH)}             & \TalentMenschenkenntnisTaWL   & \TalentMenschenkenntnisTaWR   && \multicolumn{2}{l|}{\TalentHandwerkExtraD\ \TalentHandwerkExtraDProbe} & \TalentHandwerkExtraDTaWL  & \TalentHandwerkExtraDTaWR\\\hline
& \multicolumn{4}{l|}{Überreden (MU $\bullet$ IN $\bullet$ CH)}                    & \TalentUeberredenTaWL         & \TalentUeberredenTaWR         && \multicolumn{2}{l|}{\TalentHandwerkExtraE\ \TalentHandwerkExtraEProbe} & \TalentHandwerkExtraETaWL  & \TalentHandwerkExtraETaWR\\\hline
& \multicolumn{4}{l|}{\TalentGesellschaftExtraA\ \TalentGesellschaftExtraAProbe} & \TalentGesellschaftExtraATaWL & \TalentGesellschaftExtraATaWR && \multicolumn{2}{l|}{\TalentHandwerkExtraF\ \TalentHandwerkExtraFProbe} & \TalentHandwerkExtraFTaWL  & \TalentHandwerkExtraFTaWR\\\hline
& \multicolumn{4}{l|}{\TalentGesellschaftExtraB\ \TalentGesellschaftExtraBProbe} & \TalentGesellschaftExtraBTaWL & \TalentGesellschaftExtraBTaWR && \multicolumn{2}{l|}{\TalentHandwerkExtraG\ \TalentHandwerkExtraGProbe} & \TalentHandwerkExtraGTaWL  & \TalentHandwerkExtraGTaWR\\\hline
& \multicolumn{4}{l|}{\TalentGesellschaftExtraC\ \TalentGesellschaftExtraCProbe} & \TalentGesellschaftExtraCTaWL & \TalentGesellschaftExtraCTaWR && \multicolumn{2}{l|}{\TalentHandwerkExtraH\ \TalentHandwerkExtraHProbe} & \TalentHandwerkExtraHTaWL  & \TalentHandwerkExtraHTaWR\\\hline
& \multicolumn{4}{l|}{\TalentGesellschaftExtraD\ \TalentGesellschaftExtraDProbe} & \TalentGesellschaftExtraDTaWL & \TalentGesellschaftExtraDTaWR && \multicolumn{2}{l|}{\TalentHandwerkExtraI\ \TalentHandwerkExtraIProbe} & \TalentHandwerkExtraITaWL  & \TalentHandwerkExtraITaWR\\\hline
& \multicolumn{4}{l|}{\TalentGesellschaftExtraE\ \TalentGesellschaftExtraEProbe} & \TalentGesellschaftExtraETaWL & \TalentGesellschaftExtraETaWR && \multicolumn{2}{l|}{\TalentHandwerkExtraJ\ \TalentHandwerkExtraJProbe} & \TalentHandwerkExtraJTaWL  & \TalentHandwerkExtraJTaWR\\\hline
\specialrule{3pt}{0pt}{0pt}
\end{tabular}
}
\vfill
{\footnotesize \footline}
 % zweite Seite (Talente)
\newpage
%
% Basiswertübersicht
\renewcommand{\arraystretch}{1.5}
\vspace*{3mm}
\hspace*{-6.3mm}
{
\begin{tabular}{!{\VRule[3pt]}p{42.5mm}p{42.5mm}p{42.5mm}p{42.5mm}!{\VRule[3pt]}}
\specialrule{3pt}{0pt}{0pt}
\textbf{Attacke-Basiswert }\BasisATaktuell & \textbf{Parade-Basiswert }\BasisPAaktuell  & \textbf{Fernkampf-Basiswert }\BasisFKaktuell  & \textbf{Initiative-Basiswert }\BasisINIaktuell\\
\specialrule{3pt}{0pt}{0pt}
\end{tabular}
}
\renewcommand{\arraystretch}{1.2}
\vspace*{2mm}
%-----------------------------------------------------------------------------
% Waffen & Kampfwerte
\begin{center}
{\Huge \textbf{Waffen \& Kampfwerte}}\\[2mm]
\end{center}
{ \small
\begin{tabular}{
		!{\VRule[3pt]}p{39mm} %Nahkampfwaffe
		|p{16mm} %Typ/eBE
		|p{8mm} %DK
		|p{7mm} %TP
		|p{14mm} %TP/KK
		|p{6mm} %INI
		|p{8mm} %WM
		|p{6mm} %AT
		|p{6mm} %PA
		|p{8mm} %TP
		|p{1mm} %Bruchfaktor
		|p{1mm}
		|p{1mm}
		|p{1mm}
		|p{1mm}
		!{\VRule[3pt]}
	}
\specialrule{3pt}{0pt}{0pt}
\textbf{Nahkampfwaffe} & \textbf{Typ/eBE} & \textbf{DK} & \textbf{TP} & \textbf{TP/KK} & \textbf{INI} & \textbf{WM} & \textbf{AT} & \textbf{PA} & \textbf{TP} & \multicolumn{5}{c!{\VRule[3pt]}}{\textbf{Bruchfaktor}}\\\specialrule{1.5pt}{0pt}{0pt}
\WaffeNahkampfA & \WaffeNahkampfATypeBE & \WaffeNahkampfADK & \WaffeNahkampfATP & \WaffeNahkampfATPKK & \WaffeNahkampfAINI & \WaffeNahkampfAWM & \WaffeNahkampfAAT & \WaffeNahkampfAPA & \WaffeNahkampfATP & \WaffeNahkampfABFa & \WaffeNahkampfABFb & \WaffeNahkampfABFc & \WaffeNahkampfABFd & \WaffeNahkampfABFe \\\hline
\WaffeNahkampfB & \WaffeNahkampfBTypeBE & \WaffeNahkampfBDK & \WaffeNahkampfBTP & \WaffeNahkampfBTPKK & \WaffeNahkampfBINI & \WaffeNahkampfBWM & \WaffeNahkampfBAT & \WaffeNahkampfBPA & \WaffeNahkampfBTP & \WaffeNahkampfBBFa & \WaffeNahkampfBBFb & \WaffeNahkampfBBFc & \WaffeNahkampfBBFd & \WaffeNahkampfBBFe \\\hline
\WaffeNahkampfC & \WaffeNahkampfCTypeBE & \WaffeNahkampfCDK & \WaffeNahkampfCTP & \WaffeNahkampfCTPKK & \WaffeNahkampfCINI & \WaffeNahkampfCWM & \WaffeNahkampfCAT & \WaffeNahkampfCPA & \WaffeNahkampfCTP & \WaffeNahkampfCBFa & \WaffeNahkampfCBFb & \WaffeNahkampfCBFc & \WaffeNahkampfCBFd & \WaffeNahkampfCBFe \\\hline
\WaffeNahkampfD & \WaffeNahkampfDTypeBE & \WaffeNahkampfDDK & \WaffeNahkampfDTP & \WaffeNahkampfDTPKK & \WaffeNahkampfDINI & \WaffeNahkampfDWM & \WaffeNahkampfDAT & \WaffeNahkampfDPA & \WaffeNahkampfDTP & \WaffeNahkampfDBFa & \WaffeNahkampfDBFb & \WaffeNahkampfDBFc & \WaffeNahkampfDBFd & \WaffeNahkampfDBFe \\\hline
\WaffeNahkampfE & \WaffeNahkampfETypeBE & \WaffeNahkampfEDK & \WaffeNahkampfETP & \WaffeNahkampfETPKK & \WaffeNahkampfEINI & \WaffeNahkampfEWM & \WaffeNahkampfEAT & \WaffeNahkampfEPA & \WaffeNahkampfETP & \WaffeNahkampfEBFa & \WaffeNahkampfEBFb & \WaffeNahkampfEBFc & \WaffeNahkampfEBFd & \WaffeNahkampfEBFe \\\specialrule{1.5pt}{0pt}{0pt}
\multicolumn{15}{!{\VRule[3pt]}l!{\VRule[3pt]}}{\textbf{Sonderfertigkeiten:} \SonderfertigkeitenNahkampfA}\\\hline
\multicolumn{15}{!{\VRule[3pt]}l!{\VRule[3pt]}}{\SonderfertigkeitenNahkampfB}\\\hline
\multicolumn{15}{!{\VRule[3pt]}l!{\VRule[3pt]}}{\SonderfertigkeitenNahkampfC}\\
\specialrule{3pt}{0pt}{0pt}
\end{tabular}
\\[3mm]
\begin{tabular}{
		!{\VRule[3pt]}p{37mm} %Fernkampfwaffe
		|p{16mm} %Typ/eBE
		|p{7mm} %TP
		|p{22mm} %Entfernungen
		|p{26mm} %TP/Entfernung
		|p{6mm} %FK
		|p{1mm} %Anzahl Geschosse
		|p{1mm}
		|p{1mm}
		|p{1mm}
		|p{1mm}
		|p{1mm}
		|p{1mm}
		|p{1mm}
		|p{1mm}
		!{\VRule[3pt]}
	}
\specialrule{3pt}{0pt}{0pt}
\textbf{Fernkampfwaffe} & \textbf{Typ/eBE} & \textbf{TP} & \textbf{Entfernungen} & \textbf{TP/Entfernung} & \textbf{FK} & \multicolumn{9}{c!{\VRule[3pt]}}{\textbf{Anzahl Geschosse}}\\\specialrule{1.5pt}{0pt}{0pt}
\WaffeFernkampfA & \WaffeFernkampfATypeBE & \WaffeFernkampfATP & \WaffeFernkampfAEntfernung & \WaffeFernkampfATPEntfernung & \WaffeFernkampfAFK & \WaffeFernkampfAGeschosseA & \WaffeFernkampfAGeschosseB & \WaffeFernkampfAGeschosseC & \WaffeFernkampfAGeschosseD & \WaffeFernkampfAGeschosseE & \WaffeFernkampfAGeschosseF & \WaffeFernkampfAGeschosseG & \WaffeFernkampfAGeschosseH & \WaffeFernkampfAGeschosseI \\\hline
\WaffeFernkampfB & \WaffeFernkampfBTypeBE & \WaffeFernkampfBTP & \WaffeFernkampfBEntfernung & \WaffeFernkampfBTPEntfernung & \WaffeFernkampfBFK & \WaffeFernkampfBGeschosseA & \WaffeFernkampfBGeschosseB & \WaffeFernkampfBGeschosseC & \WaffeFernkampfBGeschosseD & \WaffeFernkampfBGeschosseE & \WaffeFernkampfBGeschosseF & \WaffeFernkampfBGeschosseG & \WaffeFernkampfBGeschosseH & \WaffeFernkampfBGeschosseI \\\hline
\WaffeFernkampfC & \WaffeFernkampfCTypeBE & \WaffeFernkampfCTP & \WaffeFernkampfCEntfernung & \WaffeFernkampfCTPEntfernung & \WaffeFernkampfCFK & \WaffeFernkampfCGeschosseA & \WaffeFernkampfCGeschosseB & \WaffeFernkampfCGeschosseC & \WaffeFernkampfCGeschosseD & \WaffeFernkampfCGeschosseE & \WaffeFernkampfCGeschosseF & \WaffeFernkampfCGeschosseG & \WaffeFernkampfCGeschosseH & \WaffeFernkampfCGeschosseI \\\specialrule{1.5pt}{0pt}{0pt}
\multicolumn{15}{!{\VRule[3pt]}l!{\VRule[3pt]}}{\textbf{Sonderfertigkeiten:} \SonderfertigkeitenFernkampfA}\\\hline
\multicolumn{15}{!{\VRule[3pt]}l!{\VRule[3pt]}}{\SonderfertigkeitenFernkampfB}\\
\specialrule{3pt}{0pt}{0pt}
\end{tabular}
\\[3mm]
\begin{tabular}{
		!{\VRule[3pt]}p{40mm} %Waffenloser Kampf
		|p{13mm} %TP/KK
		|p{11mm} %INI
		|p{11mm} %AT
		|p{11mm} %PA
		|p{14mm} %TP(A)
		!{\VRule[3pt]}
	}
\specialrule{3pt}{0pt}{0pt}
\textbf{Waffenloser Kampf} & \textbf{TP/KK} & \textbf{INI} & \textbf{AT} & \textbf{PA} & \textbf{TP(A)}\\\specialrule{1.5pt}{0pt}{0pt}
Raufen & \WaffenlosRaufenTPKK & \WaffenlosRaufenINI & \WaffenlosRaufenAT & \WaffenlosRaufenPA & \WaffenlosRaufenTPA \\\hline
Ringen & \WaffenlosRingenTPKK & \WaffenlosRingenINI & \WaffenlosRingenAT & \WaffenlosRaufenPA & \WaffenlosRaufenTPA \\\specialrule{1.5pt}{0pt}{0pt}
\multicolumn{6}{!{\VRule[3pt]}l!{\VRule[3pt]}}{\textbf{Sonderfertigkeiten/Manöver:} \SonderfertigkeitenWaffeloserKampfA}\\\hline
\multicolumn{6}{!{\VRule[3pt]}l!{\VRule[3pt]}}{\SonderfertigkeitenWaffeloserKampfB}\\\hline
\multicolumn{6}{!{\VRule[3pt]}l!{\VRule[3pt]}}{\SonderfertigkeitenWaffeloserKampfC}\\
\specialrule{3pt}{0pt}{0pt}
\end{tabular}
\\[2mm]
%-----------------------------------------------------------------------------
% Schild/Parierwaffe
{\hspace*{4cm}\Large \textbf{Schild/Parierwaffe}}\\[2mm]
\begin{tabular}{
		!{\VRule[3pt]}p{35mm} %Name
		|p{16mm} %Typ
		|p{9mm} %INI
		|p{9mm} %WM
		|p{9mm} %PA
		|p{1mm} %Bruchfaktor
		|p{1mm} %Bruchfaktor
		|p{1mm} %Bruchfaktor
		|p{1mm} %Bruchfaktor
		|p{1mm} %Bruchfaktor
		!{\VRule[3pt]}
	}
\specialrule{3pt}{0pt}{0pt}
\textbf{Name} & \textbf{Typ} & \textbf{INI} & \textbf{WM} & \textbf{PA} & \multicolumn{5}{c!{\VRule[3pt]}}{\textbf{Bruchfaktor}}\\\specialrule{1.5pt}{0pt}{0pt}
\WaffeSchildA & \WaffeSchildATyp & \WaffeSchildAINI & \WaffeSchildAWM & \WaffeSchildAPA & \WaffeSchildABFa & \WaffeSchildABFb & \WaffeSchildABFc & \WaffeSchildABFd & \WaffeSchildABFe \\\hline
\WaffeSchildB & \WaffeSchildBTyp & \WaffeSchildBINI & \WaffeSchildBWM & \WaffeSchildBPA & \WaffeSchildBBFa & \WaffeSchildBBFb & \WaffeSchildBBFc & \WaffeSchildBBFd & \WaffeSchildBBFe \\\specialrule{2pt}{0pt}{0pt}
\multicolumn{10}{!{\VRule[3pt]}l!{\VRule[3pt]}}{\SFKreuzchen{Linkhand}Linkhand (PA+1), \SFKreuzchen{Schildkampf I}Schildkampf I (PA+2)/\SFKreuzchen{Schildkampf II}II (PA+2), \SFKreuzchen{Parierwaffen I}Parierwaffen I / \SFKreuzchen{Parierwaffen II}II}\\
\specialrule{3pt}{0pt}{0pt}
\end{tabular}
\\[2mm]
%-----------------------------------------------------------------------------
% Rüstung & Ausweichen, Wunden
{\hspace*{2cm}\Large \textbf{Rüstung}\hspace*{4cm}\textbf{Ausweichen}}\\[2mm]
\begin{tabular}{
		!{\VRule[3pt]}p{35mm} %Rüstungsstück
		|p{5mm} %RS
		|p{5mm} %BE
		!{\VRule[3pt]}
	}
\specialrule{3pt}{0pt}{0pt}
\textbf{Rüstungsstück}   & \textbf{RS}   & \textbf{BE}\\\hline
\RuestungA & \RuestungARS & \RuestungABE \\\hline
\RuestungB & \RuestungBRS & \RuestungBBE \\\hline
\RuestungC & \RuestungCRS & \RuestungCBE \\\hline
\RuestungD & \RuestungDRS & \RuestungDBE \\\hline
\RuestungE & \RuestungERS & \RuestungEBE \\\hline
\textbf{Summe} & \RuestungSummeRS & \RuestungSummeBE \\\hline
\multicolumn{2}{!{\VRule[3pt]}l|}{\footnotesize Rüstungsgewöhnung \SFKreuzchen{Rüstungsgewöhnung I}I, \SFKreuzchen{Rüstungsgewöhnung II}II, \SFKreuzchen{Rüstungsgewöhnung III}III} & \RuestungsgewoehnungWert\\
\specialrule{3pt}{0pt}{0pt}
\end{tabular}
\renewcommand{\arraystretch}{0.75}
\begin{tabular}{
		!{\VRule[3pt]}@{ }>{\centering\arraybackslash}p{10mm}% @{} Redefiniert den Standard Left-Margin-Wert. Leerzeichen sind Abstand
		@{}>{\centering\arraybackslash}p{10mm} %
		@{ }p{30mm}@{} %
		>{\centering\arraybackslash}p{12mm}@{} %
		!{\VRule[3pt]}
	}
\specialrule{3pt}{0pt}{0pt}
\tiny\textbf{PABasis} & \tiny\textbf{BE} & \tiny\textbf{Sonderfertigkeiten} &  \tiny\textbf{Summe}\\
\multirow{4}{*}{\large\textbf{\BasisPAaktuell \ $-$}} & \multirow{4}{*}{\large\textbf{\RuestungSummeBE \ +}} & {\tiny\SFKreuzchen{Ausweichen~I}Ausweichen I (+3)} & \multirow{4}{*}{\large\textbf{$=$ \AusweichenSumme}}\\
&& {\tiny\SFKreuzchen{Ausweichen~II}Ausweichen II (+3)} &\\
&& {\tiny\SFKreuzchen{Ausweichen~III}Ausweichen III (+3)} &\\
&& {\tiny Akrobatik? Flink/Behäbig?} &\\
\specialrule{3pt}{0pt}{0pt}
\multicolumn{4}{c}{}\\
\multicolumn{4}{c}{\Large\textbf{Wunden}}\\[1mm]
\specialrule{3pt}{0pt}{0pt}
&&&\\
\multicolumn{4}{!{\VRule[3pt]}c!{\VRule[3pt]}}{\Large$\bigcirc$ $\bigcirc$ $\bigcirc$ $\bigcirc$ $\bigcirc$ $\bigcirc$ $\bigcirc$ $\bigcirc$ $\bigcirc$}\\
&&&\\
\multicolumn{4}{!{\VRule[3pt]}c!{\VRule[3pt]}}{\footnotesize\textbf{je Wunde AT,PA,FK,GE,INI-2; GS-1}}\\ % \footnotesize = Standardsize - 2pt
\specialrule{3pt}{0pt}{0pt}
\end{tabular}\\[2.5mm]
%-----------------------------------------------------------------------------
% Lebensenergie, Ausdauer etc.
{\hspace*{3cm}\Large\textbf{Lebensenergie, Ausdauer etc.}}\\[2mm]
\renewcommand{\arraystretch}{1.2}
\begin{tabular}{
		!{\VRule[3pt]}p{27mm} %
		|p{7mm} %
		|p{5mm} %
		|p{5mm} %
		|p{5mm} %
		|p{113mm} %
		!{\VRule[3pt]}
	}
\specialrule{3pt}{0pt}{0pt}
& max. & 1/2 & 1/3 & 1/4 & aktuell\\\hline
\textbf{Lebensenergie} & \BasisLEaktuell & \BasisLEaktuellHaelfte & \BasisLEaktuellDrittel & \BasisLEaktuellViertel &\\\hline
\textbf{Ausdauer} & \BasisAUaktuell & \BasisAUaktuellHaelfte & \BasisAUaktuellDrittel & \BasisAUaktuellViertel &\\
\specialrule{3pt}{0pt}{0pt}
\end{tabular}\\[2mm]
\begin{tabular}{
		!{\VRule[3pt]}p{27mm} %
		|p{7mm} %
		|p{141mm} %
		!{\VRule[3pt]}
	}
\specialrule{3pt}{0pt}{0pt}
& max. & aktuell\\\hline
\textbf{Astralenergie} & \BasisAEaktuell &\\\hline
\textbf{Karmaenergie} & \BasisKEaktuell &\\
\specialrule{3pt}{0pt}{0pt}
\end{tabular}\\[2mm]
\begin{tabular}{
		!{\VRule[3pt]}p{27mm} %
		|p{152mm} %
		!{\VRule[3pt]}
	}
\specialrule{3pt}{0pt}{0pt}
\textbf{Initiative} & INI-Basis - BE = \underline{\BasisINIaktuell} $\pm$ Mod. + W6 (bei SF Klingentänzer +2W6) = \underline{\ \ \ \ \ }\\
\specialrule{3pt}{0pt}{0pt}
\end{tabular}
}
\vfill
{\footnotesize \footline}
 % dritte Seite (Waffen & Kampfwerte)
\newpage
%
% Ausrüstung, Vermögen & Verbindungen
\renewcommand{\arraystretch}{1.2}
\vspace*{2mm}
\begin{center}
{\Huge \textbf{Ausrüstung, Vermögen \& Verbindungen}}
\end{center}
{ \small
\begin{tabular}{!{\VRule[3pt]}p{182.9mm}!{\VRule[3pt]}}
\specialrule{3pt}{0pt}{0pt}
\textbf{Kleidung} \KleidungA\\\hline
\KleidungB \\\hline
\KleidungC \\\hline
\KleidungD \\\hline
\KleidungE \\\hline
\KleidungF \\
\specialrule{3pt}{0pt}{0pt}
\end{tabular}
\\[2mm]
\begin{tabular}{!{\VRule[3pt]}p{50mm}|p{11mm}|p{19mm}!{\VRule[3pt]}p{50mm}|p{11mm}|p{19mm}!{\VRule[3pt]}}
\specialrule{3pt}{0pt}{0pt}\vspace{-2mm}
{\large\textbf{Ausrüstung}} & {\scriptsize\textbf{Gewicht}} & {\scriptsize\textbf{Wo getragen?}} & {\vspace{-2mm}\large\textbf{Ausrüstung}} & {\scriptsize\textbf{Gewicht}} & {\scriptsize\textbf{Wo getragen?}}\\\hline
\AusruestungLinksA & \AusruestungLinksAGewicht & \AusruestungLinksAWoGetragen & \AusruestungRechtsA & \AusruestungRechtsAGewicht & \AusruestungRechtsAWoGetragen\\\hline
\AusruestungLinksB & \AusruestungLinksBGewicht & \AusruestungLinksBWoGetragen & \AusruestungRechtsB & \AusruestungRechtsBGewicht & \AusruestungRechtsBWoGetragen\\\hline
\AusruestungLinksC & \AusruestungLinksCGewicht & \AusruestungLinksCWoGetragen & \AusruestungRechtsC & \AusruestungRechtsCGewicht & \AusruestungRechtsCWoGetragen\\\hline
\AusruestungLinksD & \AusruestungLinksDGewicht & \AusruestungLinksDWoGetragen & \AusruestungRechtsD & \AusruestungRechtsDGewicht & \AusruestungRechtsDWoGetragen\\\hline
\AusruestungLinksE & \AusruestungLinksEGewicht & \AusruestungLinksEWoGetragen & \AusruestungRechtsE & \AusruestungRechtsEGewicht & \AusruestungRechtsEWoGetragen\\\hline
\AusruestungLinksF & \AusruestungLinksFGewicht & \AusruestungLinksFWoGetragen & \AusruestungRechtsF & \AusruestungRechtsFGewicht & \AusruestungRechtsFWoGetragen\\\hline
\AusruestungLinksG & \AusruestungLinksGGewicht & \AusruestungLinksGWoGetragen & \AusruestungRechtsG & \AusruestungRechtsGGewicht & \AusruestungRechtsGWoGetragen\\\hline
\AusruestungLinksH & \AusruestungLinksHGewicht & \AusruestungLinksHWoGetragen & \AusruestungRechtsH & \AusruestungRechtsHGewicht & \AusruestungRechtsHWoGetragen\\\hline
\AusruestungLinksI & \AusruestungLinksIGewicht & \AusruestungLinksIWoGetragen & \AusruestungRechtsI & \AusruestungRechtsIGewicht & \AusruestungRechtsIWoGetragen\\\hline
\AusruestungLinksJ & \AusruestungLinksJGewicht & \AusruestungLinksJWoGetragen & \AusruestungRechtsJ & \AusruestungRechtsJGewicht & \AusruestungRechtsJWoGetragen\\\hline
\AusruestungLinksK & \AusruestungLinksKGewicht & \AusruestungLinksKWoGetragen & \AusruestungRechtsK & \AusruestungRechtsKGewicht & \AusruestungRechtsKWoGetragen\\\hline
\AusruestungLinksL & \AusruestungLinksLGewicht & \AusruestungLinksLWoGetragen & \AusruestungRechtsL & \AusruestungRechtsLGewicht & \AusruestungRechtsLWoGetragen\\\hline
\AusruestungLinksM & \AusruestungLinksMGewicht & \AusruestungLinksMWoGetragen & \AusruestungRechtsM & \AusruestungRechtsMGewicht & \AusruestungRechtsMWoGetragen\\\hline
\AusruestungLinksN & \AusruestungLinksNGewicht & \AusruestungLinksNWoGetragen & \AusruestungRechtsN & \AusruestungRechtsNGewicht & \AusruestungRechtsNWoGetragen\\\hline
\AusruestungLinksO & \AusruestungLinksOGewicht & \AusruestungLinksOWoGetragen & \AusruestungRechtsO & \AusruestungRechtsOGewicht & \AusruestungRechtsOWoGetragen\\\hline
\AusruestungLinksP & \AusruestungLinksPGewicht & \AusruestungLinksPWoGetragen & \AusruestungRechtsP & \AusruestungRechtsPGewicht & \AusruestungRechtsPWoGetragen\\\hline
\AusruestungLinksQ & \AusruestungLinksQGewicht & \AusruestungLinksQWoGetragen & \AusruestungRechtsQ & \AusruestungRechtsQGewicht & \AusruestungRechtsQWoGetragen\\\hline
\AusruestungLinksR & \AusruestungLinksRGewicht & \AusruestungLinksRWoGetragen & \AusruestungRechtsR & \AusruestungRechtsRGewicht & \AusruestungRechtsRWoGetragen\\
\specialrule{3pt}{0pt}{0pt}
\end{tabular}
\\[2mm]
\begin{tabular}{!{\VRule[3pt]}p{50mm}|p{5.3mm}|p{5.3mm}|p{5.3mm}|p{5.3mm}!{\VRule[3pt]}p{50mm}|p{5.3mm}|p{5.3mm}|p{5.3mm}|p{5.3mm}!{\VRule[3pt]}}
\specialrule{3pt}{0pt}{0pt}\vspace{-2mm}
{\large\textbf{Proviant/Tränke}} & \multicolumn{4}{|c!{\VRule[3pt]}}{\textbf{Rationen/Ladungen}} & {\vspace{-2mm}\large\textbf{Proviant/Tränke}} & \multicolumn{4}{|c!{\VRule[3pt]}}{\textbf{Rationen/Ladungen}}\\\hline
\ProviantLinksA & \ProviantLinksARationen & & & & \ProviantRechtsA  & \ProviantRechtsARationen & & & \\\hline
\ProviantLinksB & \ProviantLinksBRationen & & & & \ProviantRechtsB  & \ProviantRechtsBRationen & & & \\\hline
\ProviantLinksC & \ProviantLinksCRationen & & & & \ProviantRechtsC  & \ProviantRechtsCRationen & & & \\\hline
\ProviantLinksD & \ProviantLinksDRationen & & & & \ProviantRechtsD  & \ProviantRechtsDRationen & & & \\
\specialrule{3pt}{0pt}{0pt}
\end{tabular}
\\[2mm]
\begin{tabular}{!{\VRule[3pt]}p{21mm}|p{3mm}|p{3mm}|p{3mm}|p{3mm}|p{3mm}|p{3mm}|p{3mm}|p{3mm}|p{3mm}!{\VRule[3pt]}}
\specialrule{3pt}{0pt}{0pt}\vspace{-2mm}
\textbf{\large Vermögen} & \multicolumn{9}{c!{\VRule[3pt]}}{\tiny(1 Dukaten = 10 Silbertaler = 100 Heller = 1000 Kreuzer)}\\\hline
\textbf{Dukaten} & \VermoegenDukaten &&&&&&&&\\\hline
\textbf{Silbertaler} & \VermoegenSilbertaler &&&&&&&&\\\hline
\textbf{Heller} & \VermoegenHeller &&&&&&&&\\\hline
\textbf{Kreuzer} & \VermoegenKreuzer &&&&&&&&\\\hline
\multicolumn{10}{!{\VRule[3pt]}l!{\VRule[3pt]}}{\textbf{Edelsteine u.ä.}}\\\hline
\multicolumn{10}{!{\VRule[3pt]}l!{\VRule[3pt]}}{\textbf{Sonstiger Besitz}}\\\hline
\multicolumn{10}{!{\VRule[3pt]}l!{\VRule[3pt]}}{}\\\hline
\multicolumn{10}{!{\VRule[3pt]}l!{\VRule[3pt]}}{}\\
\specialrule{3pt}{0pt}{0pt}
\end{tabular}
\begin{tabular}{!{\VRule[3pt]}p{88mm}!{\VRule[3pt]}}
\specialrule{3pt}{0pt}{0pt}\vspace{-2mm}
{\large\textbf{Verbindungen}} \\\hline
\\\hline
\\\hline
\\\hline
\\\hline
\\\hline
\\\hline
\\\hline
\\
\specialrule{3pt}{0pt}{0pt}
\end{tabular}
\\[2mm]
\begin{tabular}{!{\VRule[3pt]}p{182.9mm}!{\VRule[3pt]}}
\specialrule{3pt}{0pt}{0pt}\vspace{-2mm}
\textbf{\large Notizen} \NotizA\\\hline
\NotizB\\\hline
\NotizC\\\hline
\NotizD\\\hline
\specialrule{3pt}{0pt}{0pt}
\end{tabular}
}
\vfill
{\footnotesize \footline}
 % vierte Seite (Ausrüstung, Vermögen & Verbindungen)
\ifnum\value{ZauberAnzahl}>0
	\newpage
	\begin{landscape}
%
% Zauberdokument 1
\renewcommand{\arraystretch}{1.2}
\begin{center}
{\huge \textbf{Zauberdokument 1}}
\\[2mm]
{\large
\begin{tabular}{
		!{\VRule[3pt]}p{26.5mm}
		p{26.5mm}
		p{26.5mm}
		p{26.5mm}
		p{26.5mm}
		p{26.5mm}
		p{26.5mm}
		p{26.5mm}
		p{26.5mm}
		!{\VRule[3pt]}
	}
\specialrule{3pt}{0pt}{0pt}
\multicolumn{3}{!{\VRule[3pt]}l}{\textbf{Name}} \CharakterName & \multicolumn{2}{l}{\textbf{Rasse}} \CharakterRasse & \multicolumn{2}{l}{\textbf{Kultur}} \CharakterKultur & \multicolumn{2}{l!{\VRule[3pt]}}{\textbf{Profession}} \CharakterProfession\\\hline
\multicolumn{3}{!{\VRule[3pt]}l}{\textbf{Repräsentation} \CharakterZauberRepraesentation} & \multicolumn{2}{l}{\textbf{Merkmale} \CharakterZauberMerkmale} & \multicolumn{2}{l}{\textbf{Begabungen} \CharakterZauberBegabungen} & \multicolumn{2}{l!{\VRule[3pt]}}{\textbf{Unfähigkeiten} \CharakterZauberUnfaehigkeiten}\\\hline
\textbf{MU} \EigMUaktuell & \textbf{KL} \EigKLaktuell & \textbf{IN} \EigINaktuell & \textbf{CH} \EigCHaktuell & \textbf{FF} \EigFFaktuell & \textbf{GE} \EigGEaktuell & \textbf{KO} \EigKOaktuell & \textbf{KK} \EigKKaktuell & \textbf{MR} \BasisMRaktuell\\
\specialrule{3pt}{0pt}{0pt}
\end{tabular}\\[5mm]
\footnotesize
\begin{tabular}{
		!{\VRule[3pt]}p{4mm} % SE
		|p{37mm} % Zaubername
		|p{20mm} % Probe
		|p{3.5mm} % ZfW 1
		|p{3.5mm} % ZfW 2
		|p{10mm} % ZD
		|p{26mm} % Kosten
		|p{26mm} % Reichweite
		|p{25mm} % WD
		|p{25mm} % Anmerkungen
		|p{9.1mm} % Komplexität
		|p{10mm} % Repräsentation
		|p{9.5mm} % Hauszauber?
		|p{7mm} % Lern
		!{\VRule[3pt]}
	}
\specialrule{3pt}{0pt}{0pt}
\textbf{SE} & \textbf{Zaubername} & \textbf{Probe} & \multicolumn{2}{c|}{\textbf{ZfW}} & \textbf{ZD} & \textbf{Kosten} & \textbf{Reichweite} & \textbf{WD} & \textbf{Anmerkungen} & \textbf{Komp} & \textbf{Rep} & \textbf{Haus?} & \textbf{Lern}\\\hline
\ZauberTabelleInhaltA
\specialrule{3pt}{0pt}{0pt}
\end{tabular}
}
\end{center}
\end{landscape} % fünfte Seite (Zauberdokument 1)
\fi
\ifnum\value{ZauberAnzahl}>33
	\newpage
	\begin{landscape}
%
% Zauberdokument 1
\renewcommand{\arraystretch}{1.2}
\begin{center}
{\huge \textbf{Zauberdokument 1}}
\\[2mm]
{\large
\begin{tabular}{
		!{\VRule[3pt]}p{26.5mm}
		p{26.5mm}
		p{26.5mm}
		p{26.5mm}
		p{26.5mm}
		p{26.5mm}
		p{26.5mm}
		p{26.5mm}
		p{26.5mm}
		!{\VRule[3pt]}
	}
\specialrule{3pt}{0pt}{0pt}
\multicolumn{3}{!{\VRule[3pt]}l}{\textbf{Name}} \CharakterName & \multicolumn{2}{l}{\textbf{Rasse}} \CharakterRasse & \multicolumn{2}{l}{\textbf{Kultur}} \CharakterKultur & \multicolumn{2}{l!{\VRule[3pt]}}{\textbf{Profession}} \CharakterProfession\\\hline
\multicolumn{3}{!{\VRule[3pt]}l}{\textbf{Repräsentation} \CharakterZauberRepraesentation} & \multicolumn{2}{l}{\textbf{Merkmale} \CharakterZauberMerkmale} & \multicolumn{2}{l}{\textbf{Begabungen} \CharakterZauberBegabungen} & \multicolumn{2}{l!{\VRule[3pt]}}{\textbf{Unfähigkeiten} \CharakterZauberUnfaehigkeiten}\\\hline
\textbf{MU} \EigMUaktuell & \textbf{KL} \EigKLaktuell & \textbf{IN} \EigINaktuell & \textbf{CH} \EigCHaktuell & \textbf{FF} \EigFFaktuell & \textbf{GE} \EigGEaktuell & \textbf{KO} \EigKOaktuell & \textbf{KK} \EigKKaktuell & \textbf{MR} \BasisMRaktuell\\
\specialrule{3pt}{0pt}{0pt}
\end{tabular}\\[5mm]
\footnotesize
\begin{tabular}{
		!{\VRule[3pt]}p{4mm} % SE
		|p{37mm} % Zaubername
		|p{20mm} % Probe
		|p{3.5mm} % ZfW 1
		|p{3.5mm} % ZfW 2
		|p{10mm} % ZD
		|p{26mm} % Kosten
		|p{26mm} % Reichweite
		|p{25mm} % WD
		|p{25mm} % Anmerkungen
		|p{9.1mm} % Komplexität
		|p{10mm} % Repräsentation
		|p{9.5mm} % Hauszauber?
		|p{7mm} % Lern
		!{\VRule[3pt]}
	}
\specialrule{3pt}{0pt}{0pt}
\textbf{SE} & \textbf{Zaubername} & \textbf{Probe} & \multicolumn{2}{c|}{\textbf{ZfW}} & \textbf{ZD} & \textbf{Kosten} & \textbf{Reichweite} & \textbf{WD} & \textbf{Anmerkungen} & \textbf{Komp} & \textbf{Rep} & \textbf{Haus?} & \textbf{Lern}\\\hline
\ZauberTabelleInhaltA
\specialrule{3pt}{0pt}{0pt}
\end{tabular}
}
\end{center}
\end{landscape} % sechste Seite (Zauberdokument 2)
\fi
\end{document}
